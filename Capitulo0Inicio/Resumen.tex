En este proyecto se plantea realizar la medición de parámetros para el análisis de multifractalidad y robustez en redes complejas mediante el uso de diferentes técnicas de inteligencia artificial. El objetivo del trabajo, es utilizar diferentes técnicas de inteligencia artificial para encontrar patrones o estrategias de búsqueda dentro del proceso de cálculo de las medidas, buscando reducir el espacio búsqueda de soluciones, entendiendo que la búsqueda en dicho espacio es un problema NP-Hard [42]. Finalmente, se desarrollará una librería que se pueda integrar a herramientas enfocadas en el estudio de redes complejas.
\\\\
El análisis de fractalidad y multifractalidad en redes complejas es útil en ciertas áreas del conocimiento, por ejemplo, en la astronomía se han encontrado patrones multifractales en algunos fenómenos físicos como el viento solar[26], o en las ciencias sociales donde se han
encontrado evidencia de estructuras multifractales en el comportamiento de los mercados[7].
\\\\
Las redes complejas que modelan sistemas reales suelen ser de gran tamaño[1] y los algoritmos existentes requieren realizar una o varias iteraciones en las redes para hallar estas mediciones[42], por ello en este trabajo, se estudiarán estrategias de búsqueda automática de
patrones o regularidades en dicho proceso con el fin de reducir el tiempo de cómputo
\\\\
En la literatura, se encuentran técnicas para la medición de la fractalidad, multifractalidad y robustez en redes complejas, como las técnicas de Box Covering[39] para fractalidad y las basadas parámetros en redes complejas para la robustez[28]. Por lo tanto, se plantea si desarrollar algoritmos basados en técnicas de inteligencia artificial para la búsqueda e identificación automática de patrones, permite realizar un cálculo eficiente de las mediciones de multifractalidad y robustez.



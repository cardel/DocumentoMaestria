El problema de la gestión del espectro consiste en buscar la mejor asignación posible de canales en una banda, para un grupo de operadores satisfaciendo restricciones que son impuestas por legislaciones, condiciones ambientales, estándares de tecnologías y regularizaciones existentes. Para determinar la mejor solución en éste trabajo de grado se ha definido algunos parámetros de costos por solución, los cuales son extrapolados de los análisis económicos que se han realizado a la gestión del espectro.
\\\\
La fundamentación del problema, los objetivos del proyectos y los antecedentes, se describen en el capítulo 1. En el capítulo 2 se realiza una introducción a los conceptos teóricos necesarios para soportar la solución implementada en este proyecto de grado.
\\\\ 
La implementación de la solución está basada en un modelo matemático lineal que se describe en el capítulo 4, el cuál se consigue al analizar las diferentes restricciones que se han elegido dentro de los alcances del proyecto que son definidos en el capítulo \ref{cap3}.
\\\\
El proceso de implementación es descrito en los capítulos 5 y 6, donde el aplicativo del proyecto es implementado en lenguaje Mozart, debido a la facilidad de creación de estrategias de búsqueda, diseño de estrategias de distribución y algoritmos de búsqueda que posee de forma nativa. También se realiza un proceso de implementación utilizando un algoritmo genético que se describe en el capítulo 7, en este proceso se aprovechan algunos aspectos descritos en los capítulos anteriores.
\\\\
En el capítulo 8 se explica el proceso creación del servicio Web, en el cúal se utiliza la metodología definida por el grupo Avispa\cite{Metodologia} en donde se indica los pasos necesarios para convertir una aplicación de programación por restricciones en un servicio Web.
\\\\
Para mostrar las ventajas de la programación por restricciones como método de solución del problema se realizan análisis sobre distintos escenarios que se pueden presentar en la gestión del espectro, esto se puede consultar en el capítulo \ref{capexp}.

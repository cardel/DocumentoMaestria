\section{Análisis de robustez}

La medición de la robustez se utiliza el índice R presentado por Schneider\cite{Schneider2011} para calcular la robustez en la red. En el cual se toman dos medidas

\begin{itemize}
    \item El número de nodos con respecto al total inicial de nodos del componente gigante a medida que la red pierde nodos
    \item El promedio de los caminos más cortos del componente gigante a medida que la red pierde nodos
\end{itemize}

El algoritmo de robustez se describe de la siguiente forma:

\fbox{\begin{minipage}{16.5cm}
\begin{enumerate}
    \item Se calculan las medidas iniciales en la red sin quitar ningún nodo.
    \item Se van quitando simultáneamente 10\% de los nodos de la red bajo alguna estrategia de ataque elegida.
    \item Se calcula la medida de robustez como la suma de los medidas a medida que se van quitando los nodos simultáneamente.
\end{enumerate}
\end{minipage}}

Es de anotar que en este caso se utiliza la medida de ataque simultaneo y no secuencial debido a que se debe realizar el análisis multifractal cada vez y debido a su elevado costo computacional, no es factible realizar las pruebas repitiéndolo muchas veces.

\section{Estrategias de ataque}


\subsubsection{Por grado}

Se eliminan los nodos de acuerdo a su grado.

\subsubsection{Por centralidad}

Se eliminan los nodos de acuerdo a su centralidad. Con esto se busca destruir la capacidad de transmitir información de la red.
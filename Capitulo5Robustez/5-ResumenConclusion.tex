\section{Resumen y conclusiones del capítulo}

La robustez es una medida que permite analizar cómo una red conserva su estructura a medida que pierde nodos

\begin{itemize}
    \item Para las redes libres de escala, se encuentra que tienen una gran tolerancia a fallos aleatorios, sin embargo, si sus hubs se ven afectados, la red pierde rapidamente sus propiedades estructurales
    \item Las redes de mundo pequeño conservan sus propiedades en la mayoría de ataques, excepto el de centralidad ya que su estructura está basada en que la distancia promedio entre cada par de nodos sea la menor posible, lo que implica que muchos caminos pasan por un número determinado de nodos
    \item En redes aleatorias, la robustez depende de su estructura y puede variar de acuerdo cómo es la estructura interna de la red.
\end{itemize}

Considerando la eficiencia de daño de acuerdo a las dos medidas utilizadas, las estrategias evolutiva y de recocido simulado, presentan un comportamiento mejor que el caso aleatorio, pero peor que los ataques por centralidad o grado. Esto se debe a que estas medidas están pensadas para destruir los centros de las cajas para el análisis multifractal.
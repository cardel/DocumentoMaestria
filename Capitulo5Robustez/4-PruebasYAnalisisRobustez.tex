\section{Pruebas y análisis de robustez en diferentes redes}

Las redes que se van a estudiar son:

\begin{enumerate}
    \item Red A: Red libre de escala 8000 nodos
    \item Red B: Red de mundo pequeño 5000 nodos, probabilidad de reconexión 10\%
    \item Red C: Red aleatoria 1991 nodos
    \item Red D: Red obtenida de bacteria C.elegans
    \item Red E: Red fractal (1,3)-flower
\end{enumerate}

\subsection{Pruebas en red libre de escala}

\begin{figure}[H]
    \centering
    \includegraphics[scale=0.7]{Capitulo5Robustez/imagenes/grafica_GC20180512_143117ScaleFree8000Nodes.png}
    \caption{Medida componente de gigante}
\end{figure}


\begin{figure}[H]
    \centering
    \includegraphics[scale=0.7]{Capitulo5Robustez/imagenes/grafica_APL20180512_143117ScaleFree8000Nodes.png}
    \caption{Medida componente de promedio de camino más cortos}
\end{figure}

En las redes libres de escala, las medidas indican que la red pierde sus propiedades estructurales más rápidamente con los ataques de grado y centralidad. Esto se debe a que la pérdida hubs, implica que la red pierde capacidad de transmitir información y esto se refleja en que los caminos más cortos entre cualquier par de nodos, tienden a ser más largos.

Los ataques genético y recocido presentan una efectividad intermedia de afectación de las propiedades estructurales de la red, son mejores que el caso aleatorio, pero en las pruebas realizadas no presentan mejores resultados que los ataques por grado y centralidad.

\subsection{Pruebas red de mundo pequeño}

\begin{figure}[H]
    \centering
    \includegraphics[scale=0.7]{Capitulo5Robustez/imagenes/grafica_GC20180510_143549SmallWorld5000NodesRewire01.png}
    \caption{Medida componente de gigante}
    \label{figure:smallWorldGC}
\end{figure}


\begin{figure}[H]
    \centering
    \includegraphics[scale=0.7]{Capitulo5Robustez/imagenes/grafica_APL20180510_143549SmallWorld5000NodesRewire01.png}
    \caption{Medida componente de promedio de camino más cortos}
    \label{figure:smallWorldAPL}
\end{figure}

En el caso de las redes de mundo pequeño, la medida que cobra importancia es el promedio de los caminos más cortos, ya que al destruir nodos por centralidad se pierde la propiedad de conservar un valor promedio pequeño de distancia más corta entre un par de nodos cualesquiera.


\subsection{Pruebas red aleatoria}


\begin{figure}[H]
    \centering
    \includegraphics[scale=0.7]{Capitulo5Robustez/imagenes/grafica_GC20180501_072543Random1991Nodes5939.png}
    \caption{Medida componente de gigante}
\end{figure}


\begin{figure}[H]
    \centering
    \includegraphics[scale=0.7]{Capitulo5Robustez/imagenes/grafica_APL20180501_072543Random1991Nodes5939.png}
    \caption{Medida componente de promedio de camino más cortos}
\end{figure}

En las redes aleatorias, el ataque por centralidad tiene el mismo efecto que las redes de mundo pequeño.


\subsection{Prueba en red observada Celengs}

\begin{figure}[H]
    \centering
    \includegraphics[scale=0.7]{Capitulo5Robustez/imagenes/grafica_GC20180508_020345Celengs.png}
    \caption{Medida componente de gigante}
\end{figure}


\begin{figure}[H]
    \centering
    \includegraphics[scale=0.7]{Capitulo5Robustez/imagenes/grafica_APL20180508_020345Celengs.png}
    \caption{Medida componente de promedio de camino más cortos}
\end{figure}


Este resultado indica que al eliminar nodos de acuerdo a su grado, el tamaño del componente gigante se ve reducido considerablemente.

\subsection{Prueba red fractal}

\begin{figure}[H]
    \centering
    \includegraphics[scale=0.7]{Capitulo5Robustez/imagenes/grafica_GC20180501_151350floweru1v3.png}
    \caption{Medida componente de gigante}
\end{figure}


\begin{figure}[H]
    \centering
    \includegraphics[scale=0.7]{Capitulo5Robustez/imagenes/grafica_APL20180501_151350floweru1v3.png}
    \caption{Medida componente de promedio de camino más cortos}
\end{figure}

En el caso de la red fractal, la estrategia de ataque más eficiente es la de centralidad. Esto se explica debido a que los fractales se generan iterativamente a partir de los nodos de las generaciones anteriores, los cuales pasan a ser parte de los caminos más cortos entre cualquier par de nodos.
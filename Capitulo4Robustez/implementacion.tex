Para realizar el proceso de implementación del proyecto se deben tomar en cuenta los pasos especificados en la metodología para aplicaciones Web del grupo Avispa \cite{Metodologia}.
\\\\
En este capítulo se dan las pautas más relevantes para la implementación de la aplicación que son:

\begin{itemize}
	\item Formatos de entradas y salidas.
	\item Parámetros de las aplicaciones.
\end{itemize}

Para facilitar la divulgación del proyecto, la aplicación se nombra cómo: \textbf{CaFeSA} que es la abreviación de \textit{Constraints Application For Enhanced Spectrum Allocation}, en español, aplicación por restricciones para asignación de espectro aumentado.


\section{Resumen y conclusiones del capítulo}

En este capítulo se ha obtenido un resultado interesante de este trabajo, al observar cómo varía la estructura de una red a medida que pierde nodos. En todos los casos la dimensión fractal tiende a 1 a medida que se pierden nodos, lo que es un indicador que la red pierde sus estructuras internas.

Así mismo, por tipo de red se encuentra:

\begin{enumerate}
    \item En las redes libres de escala, el ataque por grado es el que mayor tiene efecto, en la mayor parte de las pruebas con sólo perder 20\% de los nodos de mayor grado, la red pasa a ser monofractal
    \item En las redes de mundo pequeño, la estructura de la red se ve mayormente afectada por la pérdida de los nodos con mayor centralidad, esto se debe a que son fundamentales para conservar que la distancias promedio entre los nodos sea baja
\end{enumerate}

Se encuentra que en general, la estrategia más efectiva para afectar la estructura de las redes es el ataque por grado. Las redes más sensibles a este tipo de ataque son las redes libres de escala.


\section{Análisis multifractal}

\subsection{Relación entre multifractalidad y robustez}

El análisis multifractal provee una herramienta para examinar la estructura de una red. Ahora, se va hacer una relación entre esa medida y algunas redes. Las estrategias de ataque fueron definidas en el capitulo \ref{cap5}.

Las redes de estudio son:

\begin{enumerate}
    \item Red A: Red libre de escala 8000 nodos
    \item Red B: Red de mundo pequeño 5000 nodos, probabilidad de reconexión 10\%
    \item Red C: Red aleatoria 1991 nodos
    \item Red D: Red real bacteria C.elegans
    \item Red E: Red fractal (1,3)-flower
\end{enumerate}

Debido a que se debe ejecutar el análisis de multifractalidad a medida que se pierde un porcentaje de nodos, se utiliza el algoritmo SandBox.

\subsubsection{Red libre de escala}

\begin{figure}[H]
    \centering
    \includegraphics[scale=0.7]{Capitulo6MultifractalidadYRobustez/imagenes/grafica_DqRandom20180512_143117ScaleFree8000Nodes.png}
    \caption{Análisis de multifractalidad de red libre de escala a ataque aleatorio }
\end{figure}

\begin{figure}[H]
    \centering
    \includegraphics[scale=0.7]{Capitulo6MultifractalidadYRobustez/imagenes/grafica_DqDegree20180512_143117ScaleFree8000Nodes.png}
    \caption{Análisis de multifractalidad de red libre de escala a ataque por grado }
\end{figure}

\begin{figure}[H]
    \centering
    \includegraphics[scale=0.7]{Capitulo6MultifractalidadYRobustez/imagenes/grafica_DqCentrality20180512_143117ScaleFree8000Nodes.png}
    \caption{Análisis de multifractalidad de red libre de escala a ataque por centralidad }
\end{figure}


\begin{figure}[H]
    \centering
    \includegraphics[scale=0.7]{Capitulo6MultifractalidadYRobustez/imagenes/grafica_DqGenetic20180512_143117ScaleFree8000Nodes.png}
    \caption{Análisis de multifractalidad de red libre de escala a ataque por estrategia evolutiva }
\end{figure}

\begin{figure}[H]
    \centering
    \includegraphics[scale=0.7]{Capitulo6MultifractalidadYRobustez/imagenes/grafica_DqSimulated20180512_143117ScaleFree8000Nodes.png}
    \caption{Análisis de multifractalidad de red libre de escala a ataque por estrategia recocida simulada }
\end{figure}


\subsubsection{Red de mundo pequeño}
\begin{figure}[H]
    \centering
    \includegraphics[scale=0.7]{Capitulo6MultifractalidadYRobustez/imagenes/grafica_DqRandom20180510_143549SmallWorld5000NodesRewire01.png}
    \caption{Análisis de multifractalidad de red de mundo pequeño ataque aleatorio }
\end{figure}

\begin{figure}[H]
    \centering
    \includegraphics[scale=0.7]{Capitulo6MultifractalidadYRobustez/imagenes/grafica_DqDegree20180510_143549SmallWorld5000NodesRewire01.png}
    \caption{Análisis de multifractalidad de rred de mundo pequeño a ataque por grado }
\end{figure}

\begin{figure}[H]
    \centering
    \includegraphics[scale=0.7]{Capitulo6MultifractalidadYRobustez/imagenes/grafica_DqCentrality20180510_143549SmallWorld5000NodesRewire01.png}
    \caption{Análisis de multifractalidad de red de mundo pequeño a ataque por centralidad }
\end{figure}


\begin{figure}[H]
    \centering
    \includegraphics[scale=0.7]{Capitulo6MultifractalidadYRobustez/imagenes/grafica_DqGenetic20180510_143549SmallWorld5000NodesRewire01.png}
    \caption{Análisis de multifractalidad de red red de mundo pequeño a ataque por estrategia evolutiva }
\end{figure}

\begin{figure}[H]
    \centering
    \includegraphics[scale=0.7]{Capitulo6MultifractalidadYRobustez/imagenes/grafica_DqSimulated20180510_143549SmallWorld5000NodesRewire01.png}
    \caption{Análisis de multifractalidad de red de mundo pequeño a ataque por estrategia recocida simulada }
\end{figure}

\subsubsection{Red aleatoria}
\begin{figure}[H]
    \centering
    \includegraphics[scale=0.7]{Capitulo6MultifractalidadYRobustez/imagenes/grafica_DqRandom20180501_072543Random1991Nodes5939.png}
    \caption{Análisis de multifractalidad de red aleatoria a ataque aleatorio }
\end{figure}

\begin{figure}[H]
    \centering
    \includegraphics[scale=0.7]{Capitulo6MultifractalidadYRobustez/imagenes/grafica_DqDegree20180501_072543Random1991Nodes5939.png}
    \caption{Análisis de multifractalidad de red aleatoria a ataque por grado }
\end{figure}

\begin{figure}[H]
    \centering
    \includegraphics[scale=0.7]{Capitulo6MultifractalidadYRobustez/imagenes/grafica_DqCentrality20180501_072543Random1991Nodes5939.png}
    \caption{Análisis de multifractalidad de red aleatoria a ataque por centralidad }
\end{figure}


\begin{figure}[H]
    \centering
    \includegraphics[scale=0.7]{Capitulo6MultifractalidadYRobustez/imagenes/grafica_DqGenetic20180501_072543Random1991Nodes5939.png}
    \caption{Análisis de multifractalidad de red red aleatoria a ataque por estrategia evolutiva }
\end{figure}

\begin{figure}[H]
    \centering
    \includegraphics[scale=0.7]{Capitulo6MultifractalidadYRobustez/imagenes/grafica_DqSimulated20180501_072543Random1991Nodes5939.png}
    \caption{Análisis de multifractalidad de red red aleatoria a ataque por estrategia recocida simulada }
\end{figure}

\subsubsection{Red real}

\begin{figure}[H]
    \centering
    \includegraphics[scale=0.7]{Capitulo6MultifractalidadYRobustez/imagenes/grafica_DqRandom20180508_020345Celengs.png}
    \caption{Análisis de multifractalidad de red real Celegens a ataque aleatorio}
\end{figure}

\begin{figure}[H]
    \centering
    \includegraphics[scale=0.7]{Capitulo6MultifractalidadYRobustez/imagenes/grafica_DqDegree20180508_020345Celengs.png}
    \caption{Análisis de multifractalidad de red real Celegens a ataque por grado}
\end{figure}

\begin{figure}[H]
    \centering
    \includegraphics[scale=0.7]{Capitulo6MultifractalidadYRobustez/imagenes/grafica_DqCentrality20180508_020345Celengs.png}
    \caption{Análisis de multifractalidad de red real Celegens a ataque por centralidad}
\end{figure}


\begin{figure}[H]
    \centering
    \includegraphics[scale=0.7]{Capitulo6MultifractalidadYRobustez/imagenes/grafica_DqGenetic20180508_020345Celengs.png}
    \caption{Análisis de multifractalidad de red real Celegens a ataque por estrategia evolutiva}
\end{figure}

\begin{figure}[H]
    \centering
    \includegraphics[scale=0.7]{Capitulo6MultifractalidadYRobustez/imagenes/grafica_DqSimulated20180508_020345Celengs.png}
    \caption{Análisis de multifractalidad de red real Celegens a ataque por estrategia recocida simulada }
\end{figure}

\subsubsection{Red fractal}
\begin{figure}[H]
    \centering
    \includegraphics[scale=0.7]{Capitulo6MultifractalidadYRobustez/imagenes/grafica_DqRandom20180501_151350floweru1v3.png}
    \caption{Análisis de multifractalidad de red libre de escala a ataque aleatorio }
\end{figure}

\begin{figure}[H]
    \centering
    \includegraphics[scale=0.7]{Capitulo6MultifractalidadYRobustez/imagenes/grafica_DqDegree20180501_151350floweru1v3.png}
    \caption{Análisis de multifractalidad de red (1,3)-flower a ataque por grado }
\end{figure}

\begin{figure}[H]
    \centering
    \includegraphics[scale=0.7]{Capitulo6MultifractalidadYRobustez/imagenes/grafica_DqCentrality20180501_151350floweru1v3.png}
    \caption{Análisis de multifractalidad de red (1,3)-flower a ataque por centralidad }
\end{figure}


\begin{figure}[H]
    \centering
    \includegraphics[scale=0.7]{Capitulo6MultifractalidadYRobustez/imagenes/grafica_DqGenetic20180501_151350floweru1v3.png}
    \caption{Análisis de multifractalidad de red (1,3)-flower a ataque por estrategia evolutiva }
\end{figure}

\begin{figure}[H]
    \centering
    \includegraphics[scale=0.7]{Capitulo6MultifractalidadYRobustez/imagenes/grafica_DqSimulated20180501_151350floweru1v3.png}
    \caption{Análisis de multifractalidad de red (1,3)-flowera ataque por estrategia recocida simulada }
\end{figure}

\subsection{Discusión de resultados}

La relación entre multifractalidad y robustez está dada en como se comporta la dimensión fractal de la red a medida que pierde nodos.

La estrategia que más efectos produce son las estrategias de ataque por grado y por centralidad, ya que estas tienden a desintegrar la red. El efecto es tan fuerte, que eliminados un 20\% a 30\% de nodos, no es posible aplicar los algoritmos de análisis multifractal.

Las estrategias de ataque por algoritmos de Inteligencia Artificial, no tienen un efecto tan evidente como lo tienen la centralidad o el grado. Sin embargo, muestran mayor efecto en la redes que las estrategia aleatoria.

Con respecto a las redes de mundo pequeño, se observa que estas conservan su estructura a medida que pierde nodos, lo que muestra que su estructura interna permanece intacta ante diferentes estrategias de ataque.

Para las redes aleatorias, libres de escala, reales y fractales, se encuentra que pierden su estructura a medida que pierden nodos, pasando de ser objetos multifractales a monofractales. Esto se debe a que estas presentan en mayor o menor medida hubs, los cuales estructuralmente son los centros de estructuras internas, que a medida que se pierden la red se va tornando más uniforme.

En todos los casos, la dimensión fractal tiende hacia 1 a medida que la red pierde nodos. Una dimensión fractal 1, indica que la red pasa de ser un conjunto n-dimensional a uno unidimensional.
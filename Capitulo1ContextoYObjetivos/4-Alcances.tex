\section{Alcances}

\subsection{Limites computacionales}
Se debe tener en cuenta que los recursos computacionales son limitados y el problema a solucionar es NP-Hard, por lo que podrían existir limitaciones en tiempo y memoria para procesar algunas redes.

\subsection{Solución óptima}
Debido a que el uso de las técnicas de inteligencia artificial implica un proceso de aprendizaje de cómo medir parámetros, no se puede garantizar que la solución encontrada sea la
óptima, debido a que este proceso depende en gran medida de los datos que se utilizaron para
su entrenamiento.
\subsection{Librería prototipo}

La librería que se va a construir es únicamente para propósito validación de las técnicas aplicadas en esta tesis, por lo tanto, no representa un producto que se pueda utilizar comercialmente y sin conocimientos en redes complejas.
\subsection{Validez de las pruebas}

Se van a realizar pruebas en redes complejas representativas, para realizar el análisis de los algoritmos diseñados en este trabajo, sin embargo, debido a que pueden existir redes con características únicas, en las que las soluciones basadas en técnicas de inteligencia artificial podrían presentar resultados insatisfactorios, las pruebas sólo validarán el trabajo realizado en un conjunto seleccionado de redes complejas.

\section{Objetivos}

\subsection{Objetivo general}

Desarrollar una librería prototipo de medición de parámetros para el análisis de la multifractalidad y robustez en redes complejas utilizando técnicas de inteligencia artificial.

\subsection{Objetivos espec\'ificos y resultados esperados}

\begin{table}[H]
    \centering
    \begin{tabular}{|p{11cm}|p{5cm}|}
        \hline
        \textbf{Objetivo específico} & \textbf{Sección del documento} \\
        \hline
        Implementar dos algoritmos encontrados en la literatura para la medición de multifractalidad y robustez.& Capítulo \ref{cap2}, Capitulo \ref{cap4} sección \ref{cap4:seccionBox} y sección \ref{cap4:seccionSB}\\
        \hline
        Identificar dos técnicas de inteligencia artificial que se puedan utilizar para la medición de parámetros en el análisis de multifractalidad y robustez en redes complejas.&  Capítulo \ref{cap4} secciones \ref{cap4:evolutivo} y \ref{cap4:recocido}\\
        \hline
         Desarrollar e implementar funciones que
permitan realizar la medición de parámetros para el análisis de multifractalidad y robustez en redes complejas, utilizando las técnicas de inteligencia artificial elegidas. &  Capítulo \ref{cap4} secciones \ref{cap4:evolutivo} y \ref{cap4:recocido}, y Capitulo \ref{cap7}\\
        \hline
         Experimentar con los algoritmos basados
en inteligencia artificial y dos algoritmos
encontrados en la literatura.&  Capítulo \ref{cap4} sección \ref{cap4:analisisIA}, Capitulo \ref{cap6} y Capitulo \ref{cap8}
\\
        \hline
        Análisis comparativo entre las soluciones basadas en inteligencia artificial y dos algoritmos encontrados en la literatura.  & Capítulo \ref{cap4} sección \ref{cap4:analisisIA} y Capitulo \ref{cap6}\\
        \hline
    \end{tabular}
    \caption{Objetivos específicos}
    \label{tab:objEspecificos}
\end{table}








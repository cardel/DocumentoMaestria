\section{Justificación}

\subsection{Justificación teórica}

El problema de asignación del espectro es un problema combinatorio, donde se requiere realizar una asignación de canales para un grupo de operadores buscando que la asignación satisfaga unas restricciones y minimice costos. 
\\\\
Este problema es similar al de asignación de canales (\textit{channel assigment})\cite{680521} en el que se debe asignar frecuencias a un grupo de transmisores en un sistema de celdas buscando que no existan interferencias entre celdas vecinas. Con esta aproximación se busca encontrar un modelo lineal para solucionar este problema.

\subsection{Justificación metodológica}

El paradigma de programación por restricciones es muy útil en problemas donde se requiere realizar programación de tareas y planificación satisfaciendo restricciones, por lo que también parece muy apropiado para aplicar en la solución del problema que se plantea en este trabajo de grado.
\\\\
La generación de un modelo lineal que pueda representar las diferentes restricciones que están presentes en el proceso de gestión del espectro, representa una ventaja debido a que el modelo es más intuitivo y se puede usar en otros métodos de solución diferentes a la programación por restricciones.
\\\\
El uso de de la metodología definida por el grupo Avispa, para que sus aplicaciones puedan ser utilizadas como un servicio Web, es un valor agregado para futuros trabajos en ésta área, debido que permite a partir del desarrollo de éste proyecto observar las ventajas y desventajas que ésta presenta para estos desarrollos.


\subsection{Justificación práctica}

Actualmente, es de gran importancia que la gestión del espectro radioeléctrico sea óptima, ya que si se presentan problemas de asignación o interferencia en algunas zonas, se deben asumir costos adicionales, debido a cambios que deban realizarse en los equipos de transmisión o en la compra de equipos para el filtrado de señales interferidas.	
\\\\
Las tecnologías de comunicación inalámbrica, han tenido en los últimos años un gran impacto en la comunidad, ya que se han convertido en un medio indispensable para la comunicación e interacción entre las personas, por lo que es muy importante garantizar una buena calidad de servicio y facilidad de expansión de estas tecnologías con una correcta gestión del espectro radioeléctrico.
\\\\
Una correcta gestión del espectro, permite a los operadores adquirir tecnología para la transmisión y recepción de señales en una sola banda de frecuencia, además mejora los índices de calidad del servicio de comunicación inalámbrica ya que se reducen los niveles de interferencia de las señales.



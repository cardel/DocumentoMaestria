\section{Justificación}


Este trabajo busca integrar el área de inteligencia artificial (IA) con el estudio de redes complejas, las técnicas de inteligencia artificial ofrecen soluciones basadas en la identificación de patrones y aprendizaje de problemas, que permiten diseñar soluciones que realizan una búsqueda eficiente en problemas computacionalmente NP-Hard.

Los algoritmos disponibles en la literatura para la medición de multifractalidad y robustez,
como el algoritmo de Box Covering, a medida que exploran la red, requieren la evaluación
de un conjunto de soluciones, lo que implica realizar cálculos en un espacio combinatorio.
Como consecuencia, no son prácticos para redes gran tamaño, por esta razón, en este trabajo
se aborda el uso de técnicas de inteligencia artificial para encontrar patrones o reglas que
permitan plantear técnicas para realizar estas mediciones más rápidamente.

La identificación de patrones y regularidades en la medición de parámetros en redes complejas, ofrece una experiencia para el diseño de algoritmos y estrategias para el estudio de
características de diferentes sistemas complejos en diferentes áreas del conocimiento, como es
el caso de las ciencias sociales, donde las poblaciones pueden ser caracterizadas o clasificadas como comunidades a través de una medida de modularidad, o en el estudio de redes de relaciones entre personas definidas por redes de conocidos o citaciones. Otro caso, es el estudio de sistemas urbanos, como son las redes de transporte, en donde se pueden identificar estructuras que ayuden a tomar decisiones para enfrentar problemas como los atascos de tráfico.






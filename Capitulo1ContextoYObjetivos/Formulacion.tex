\section{Planteamiento del problema}
En sistemas complejos reales, tales como, las redes de interacciones entre proteínas, redes de transporte y redes sociales, las mediciones de multifractalidad y robustez son de gran importancia. La multifractalidad, por su parte, permite identificar subestructuras que son significativas en una red\cite{Shuhei2011} y, por otra parte, la robustez es una métrica para el análisis de tolerancia a fallos\cite{Martin-Hernandez2013}. Además, estas medidas también sirven de apoyo a áreas como la astronomía donde eventos como el viento solar puede ser estudiado identificado estructuras fractales\cite{Macek2007} o en el economía, donde algunos estudios evidencia la existencia multifractalidad dentro del comportamiento de los mercados\cite{Caraiani2012}.
\\\\
Por lo tanto, proveer una librería que permita realizar estas mediciones en diferentes entornos de trabajo cobra importancia, ya que estas son de gran apoyo para diferentes áreas del conocimiento. En la literatura, existen varias técnicas que mediante estrategias de exploración, permiten realizar estas mediciones. Dichas estrategias realizan una búsqueda intensiva en el espacio de soluciones de estos problemas que son NP-Hard, por lo que se requiere una gran capacidad de procesamiento y tiempo de ejecución, sobre todo en redes de gran tamaño. Como consecuencia, es necesario buscar técnicas que permitan identificar patrones para el diseño de estrategias que reduzcan el espacio de búsqueda. Por lo anterior, se propone utilizar técnicas de inteligencia artificial, pues estas permiten el diseño de soluciones basadas en el aprendizaje automático e identificación automática de patrones, con el objetivo de explorar eficientemente el espacio de soluciones.
\\\\
Con el trabajo se busca desarrollar técnicas basadas en inteligencia artificial, para la medición de la multifractalidad y robustez en redes complejas, ya que gran parte de las estrategias encontradas en la literatura son algorítmicas, basadas en la combinatoria de posibles soluciones. Las técnicas algorítmicas requieren realizar un gran número de pasos dentro de las redes complejas para obtener las mediciones y algunas son de estas redes son de gran tamaño, por lo que, en esta propuesta propone que estos pasos se pueden hacer forma eficiente o inteligente.

\section{Pregunta de investigación}

¿Cómo desarrollar un algoritmo que a través de patrones o regularidades identificadas automáticamente en el proceso de medición de multifractalidad y robustez en redes complejas, permitan llevar a cabo computaciones más económicas que las presentes en la literatura?

\section{Hipótesis}

Se pueden identificar patrones o regularidades asociados con los parámetros del análisis multifractal y de robustez en redes complejas, que permitan diseñar soluciones computacionalmente más económicas que los algoritmos disponibles en la literatura.
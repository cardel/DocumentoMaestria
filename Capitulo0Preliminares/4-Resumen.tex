La clasificación de las redes complejas se realiza a partir de sus propiedades estructurales, tales como la distribución de grado, grado de centralidad, grado de vecindad y distancia promedio de los caminos más cortos. 

Sin embargo, dado que la estructura de las redes complejas es muy variada, no es preciso caracterizarlas solamente con las medidas de propiedades estructurales. Un método que promete proveer más información es el análisis de multifractalidad, ya que permite estudiar las estructuras presentes dentro de la red, la cual puede ser homogénea o heterogénea. Además, se busca integrarlo con el análisis de robustez, para analizar cómo se afectan las estructuras internas a medida que la red pierde nodos y aristas 

En consecuencia, al realizar un estudio sobre la relación entre multifractalidad y robustez, se busca aportar una experiencia en la caracterización de las redes complejas, ya que varios autores indican que las medidas usadas actualmente no ofrecen una caracterización completa.

Además, se evidencia en la literatura las técnicas de análisis multifractal basadas en cobertura de cajas al ser estrategias de búsqueda intensiva, son computacionales intratables. Por esta razón, se propone una heurísticas para técnicas de inteligencia artificial, con el fin de optimizar la estrategia de búsqueda y así reducir los tiempos de cómputo. Para cumplir este objetivo, se propone una estrategia basada en estos algoritmos y a partir de ella se diseñan un algoritmo evolutivo y un algoritmo de recocido simulado.

Para estudiar la relación entre multifractalidad y robustez se realiza un conjunto de pruebas utilizando redes generadas libres de escala, de mundo pequeño, aleatorias, y redes de bases de datos seleccionadas. Con esto se busca analizar si existe una relación entre la multifractalidad y robustez. Así mismo, se aplican las estrategias de inteligencia artificial para comparar su tiempo de ejecución con las estrategias presentes en la literatura.

Los resultados encontrados permiten observar que en el caso de las redes de mundo pequeño, estas son monofractales o de estructura homogénea y que esta estructura se conserva a medida que estas redes pierden nodos. Así mismo, la relación entre las dimensiones fractales y la robustez, indican preliminarmente que la dimensión fractal se reduce a medida que la red pierde nodos. También, las pruebas indican que las estrategias de inteligencia artificial mejoran el tiempo requerido para el análisis multifractal ya que permiten aplicar una estrategia para la ubicación de las cajas, pues los algoritmos presentes en la literatura seleccionan los centros de las cajas de forma repetitiva, de tal forma se aproxime el cubrimiento óptimo.



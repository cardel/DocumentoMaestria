La clasificación de las redes complejas se realiza a partir de sus propiedades estructurales, tales como la distribución de grado, grado de centralidad, grado de vecindad, distancia promedio de los caminos más cortos, entre otras. 

Sin embargo, existen métodos más completos que permiten analizar la topología de una red. Un método que promete proveer más información es el análisis de multifractalidad, ya que permite estudiar estructuras que se repiten y que conforman la macro estructura de la red. Esto podría llamarse como redes de estructura homogénea donde una sola estructura compone la red, o heterogénea en el caso donde hay estructuras que se repiten y conforman en si misma una nueva estructura más compleja. Además, se busca compararlo con el análisis de robustez, para analizar cómo se afectan las estructuras internas a eliminar a medida que la red pierde nodos y aristas 

En consecuencia, al realizar un estudio sobre la relación entre multifractalidad y robustez, se busca aportar una herramienta en la caracterización de las redes complejas, esto con el objetivo de tener una imagen lo más completa posible de la complejidad de una red, ya que varios autores indican que las medidas usadas actualmente no ofrecen esta caracterización.

Además, se evidencia en la literatura las técnicas de análisis multifractal basadas en cobertura de cajas al ser estrategias de búsqueda intensiva, son computacionales intratables. Por esta razón, se propone una heurísticas para técnicas de inteligencia artificial, con el fin de optimizar la estrategia de búsqueda y así reducir los tiempos de cómputo. Para cumplir este objetivo, se propone una estrategia basada en estos algoritmos y a partir de ella se diseñan un algoritmo evolutivo y un algoritmo de recocido simulado.

Para estudiar la relación entre multifractalidad y robustez se realizan experimentos utilizando redes generadas libres de escala, de mundo pequeño, aleatorias, y redes reportadas en la literatura. Con esto se busca analizar si existe una relación entre la multifractalidad y robustez. Así mismo, se aplican las estrategias de inteligencia artificial para comparar su tiempo de ejecución con las estrategias presentes en la literatura.

Los resultados encontrados permiten observar que las redes de mundo pequeño tienden a ser monofractales o de estructura homogénea, tal como se ha reportado en la literatura. Adicionalmente, esta estructura se conserva a medida que las redes de mundo pequeño son afectadas por ataques, tales cómo la pérdida de nodos. Así mismo, la relación entre las dimensiones fractales y la robustez, indican preliminarmente que la dimensión fractal se reduce a medida que la red pierde nodos. También, las pruebas indican que las estrategias de inteligencia artificial mejoran el tiempo requerido para el análisis multifractal ya que permiten aplicar una estrategia para la ubicación de las cajas, pues los algoritmos presentes en la literatura seleccionan los centros de las cajas de forma repetitiva, de tal forma se aproxime el cubrimiento óptimo.



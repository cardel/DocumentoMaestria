La redes complejas permiten observar diferentes sistemas a través de una representación basada en grafos. En la literatura se presentan múltiples métricas que permiten caracterizar estas redes, como lo son la distribución de grado, coeficiente de agrupamiento, coeficiente de agrupación, etc. 

Dado que las redes complejas son muy variadas, el análisis de sus estructuras internas permite obtener más información que otras métricas que consideran la red como un todo. Uno es de estos es el análisis de multifractalidad que permite estudiar estructuras que se repiten y que conforman la macro estructura de la red.

El análisis multifractal consiste en realizar el cálculo de las dimensiones fractales de una red, las cuales permiten observar cómo cambia su estructura interna a medida que la red es escalada. A partir de este análisis se puede determinar que una red es monofractal o que conserva su estructura interna a medida que se escala, lo que indica que la red tiende a estar compuesta por una sola estructura. Así mismo, se puede determinar que una red es multifractal, lo que indica que una red tiende a estar compuesta por múltiples estructuras.

En diferentes trabajos realizados en la literatura se han estudiado múltiples redes a través del análisis multifractal y su relación con las diferentes métricas que consideran la red como un todo. Sin embargo, esto no permite estudiar con mayor detalle las estructuras internas de una red, por lo que la propuesta de este trabajo es relacionarlo con el análisis de robustez para observar cómo comporta la estructura de una red ante perturbaciones. 

En consecuencia, realizar un estudio sobre la relación entre multifractalidad y robustez busca aportar una herramienta en la caracterización de las redes complejas. Este estudio permite tener una imagen de cómo se comportan las estructuras internas de una red a medida que se pierden nodos y conexiones producto de perturbaciones en su estructura. Varios autores han indicado que las métricas más comunes en la literatura no permiten obtener esta caracterización.

Además, se evidencia en la literatura las técnicas de análisis multifractal basadas en cobertura de cajas son estrategias de búsqueda intensiva. lo que las convierte en computacionales intratables. Por esta razón, se proponen heurísticas para técnicas de inteligencia artificial, con el fin de optimizar la estrategia de búsqueda y así reducir los tiempos de cómputo. Para cumplir este objetivo, se propone una estrategia basada en la selección de los nodos que serán los centros de las cajas para el cubrimiento. Con esto, se diseñan un algoritmo evolutivo y un algoritmo de recocido simulado.

Metodológicamente, para el estudio de la relación entre la multifractalidad y robustez se realizan experimentos utilizando redes generadas libres de escala, de mundo pequeño, aleatorias y redes reportadas en la literatura. Estos experimentos consisten en calcular las dimensiones fractales cada vez que la red pierde un porcentaje de sus nodos producto de una perturbación. 

Los resultados encontrados permiten observar que las redes de mundo pequeño tienden a ser monofractales o de estructura homogénea, tal como se ha reportado en la literatura. Adicionalmente, esta estructura se conserva a medida que las redes de mundo pequeño son afectadas por ataques, tales cómo la pérdida de nodos. Así mismo, la relación entre las dimensiones fractales y la robustez, indican preliminarmente que la dimensión fractal se reduce a medida que la red pierde nodos. También, las pruebas indican que las estrategias de inteligencia artificial mejoran el tiempo requerido para el análisis multifractal, ya que permiten aplicar una estrategia para la ubicación de las cajas, pues los algoritmos presentes en la literatura seleccionan los centros de las cajas de forma repetitiva, de tal forma se aproxime el cubrimiento óptimo.
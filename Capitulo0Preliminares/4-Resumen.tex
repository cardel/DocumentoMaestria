Actualmente, la categorización de las redes complejas, se realiza a partir de sus propiedades estructurales, tales como, la distribución de grado, grado de centralidad, grado de vecindad y distancia promedio de los caminos más cortos. Para esto, se han determinado que las redes complejas pueden ser libres de escala, de mundo pequeño y aleatorias\cite{BarabasiNetwork}.

Sin embargo, dado que la estructura de las redes complejas es muy variada, no es preciso caracterizarlas solamente con las medidas de propiedades estructurales. Un método que promete proveer más información es el análisis de multifractalidad, la cual permite estudiar las estructuras presentes dentro de la red, la cual puede ser homogénea o heterogénea\cite{Liu2015}. Otro método, es el análisis de robustez, el cual permite analizar cómo se comportan las propiedades estructurales a medida que la red pierde nodos y aristas \cite{Martin-Hernandez2013}.

En consecuencia, al realizar un estudio entre la relación entre multifractalidad y robustez, se busca proveer una herramienta para el estudio de redes complejas reales, para las cuales las medidas con las que se cuenta actualmente no ofrecen una caracterización completa.

En la literatura, para la medición de multifractalidad se utilizan técnicas basadas en Box Covering\cite{Shuhei2011} las cuales al ser estrategias de búsqueda intensiva, son computacionales intratables. Por esta razón, se busca diseñar heurísticas y aplicar técnicas de inteligencia artificial para optimizar la estrategia de búsqueda y así reducir los tiempos de cómputo. Para cumplir este objetivo, se propone una estrategia heurística basada en los algoritmos de BoxCovering y a partir de ella se diseñan dos algoritmos de inteligencia artificial: un algoritmo evolutivo y un algoritmo de recocido simulado.

Para estudiar la relación entre multifractalidad y robustez, se realiza un conjunto de pruebas utilizando redes generadas libres de escala, de mundo pequeño, aleatorias, y redes reales seleccionadas. Con esto, se busca analizar si existe una relación entre la multifractalidad y robustez. Así mismo, se aplican las estrategias de inteligencia artificial, para comparar su tiempo de ejecución con las estrategias presentes en la literatura.

Los resultados encontrados, permiten observar que en el caso de las redes de mundo pequeño, estas son monofractales o de estructura homogénea y que esta estructura se conserva a medida que estas redes pierden nodos. También, los algoritmos de inteligencia artificial, mejoran el tiempo de ejecución de los algoritmos, ya que permiten aplicar una estrategia para la ubicación de las cajas, ya que los algoritmos determinísticos seleccionan los centros de las cajas de forma repetitiva, de tal forma se encuentre el cubrimiento mínimo.



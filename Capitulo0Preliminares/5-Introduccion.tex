Las redes complejas permiten modelar sistemas de la vida real, como los sistemas biológicos, en los cuales se pueden realizar modelos de interacción entre proteínas, de organismos en un ecosistema, entre otros. Los sistemas de transporte se modelan a partir de las interacciones entre los vehículos, los pasajeros y el sistema vial de una ciudad. Los sistemas sociales, se modelan a partir de las interacciones entre las personas, un caso de esto son los disturbios en eventos deportivos. Otro campo, son las redes sociales y de citaciones de artículos científicos, entre otras. En consecuencia, las redes complejas son una herramienta que proporciona un soporte matemático para el estudio de diversos problemas en diferentes áreas del conocimiento.

En las ciencias de la computación, las redes complejas son representadas como grafos, en los cuales los vértices representan una entidad en el modelo y las aristas las relaciones entre ellas. A partir de esta representación, se pueden estudiar algunas propiedades estructurales de las redes, como es la distribución de grado que permite visualizar cómo están conectados los vértices en la red. Las medidas de cercanía por centralidad por intermediación, que indican si un vértice hace parte de los caminos entre un par de vértices. La cercanía por vecindad, la cual, se utiliza para medir que tan cerca está del centro de la red. Otra medida de gran importancia, es la longitud los caminos promedios entre un par de vértices.

De acuerdo a sus propiedades estructurales, las redes complejas se clasifican en dos categorías, redes libres de escala y redes de mundo pequeño. Las redes libres de escala tienen la característica de tener una distribución de grado acorde a una ley de potencia, en la cual pocos vértices tienen muchas conexiones y un gran número pocas conexiones. Las redes de mundo pequeño, son aquellas que tienen como característica que la distancia más corto promedio entre un par de vértices, es proporcional al logaritmo natural del número de nodos.Las redes que no entran ninguna de estas categorías, se consideran redes aleatorias.

Sin embargo, dada la complejidad de las redes complejas, estas medidas no son suficientes para clasificar las redes complejas, ya que las redes libres de escala y de mundo pequeño están basadas en modelos de generación de redes. Dos medidas que pueden aportar más información, son el análisis multifractal, el cual permite estudiar si la estructura interna de la red es homogénea o heterogénea y la robustez, que es un análisis de robustez que permite analizar cómo la red se comporta a perturbaciones en su estructura.

Para la medida de robustez, se tienen algoritmos basados en cobertura por cajas, los cuales son utilizados en el procesamiento de imágenes para el análisis fractal, con la diferencia que en redes complejas se utiliza un sistema de referencia basado en el diámetro de la red. La idea, es observar que la dimensión fractal $L_B$ es la relación entre el diámetro de las cajas $R_B$ y número de cajas $N_B$, que está dado por $N_B=R_{B}^{L_B}0$. Sin embargo, esta medida ha demostrado no ser suficiente para el estudio de fractalidad, por ello se han desarrollado variantes que permiten realizar un análisis multifractal, el cual arroja más información sobre las estructuras de la red a medida que la red es escalada por un factor $q$. En el caso de la robustez, el análisis consiste en estudiar las propiedades estructurales de las redes a medida que esta pierde nodos o aristas, seleccionados a partir de alguna estrategia de ataque.

El estudio de una relación entre la multifractalidad y robustez, promete dar un gran marco de trabajo para el estudio de las redes complejas, ya que, permiten estudiar cómo se comporta la estructura de la red a medida que esta red pierde vértices y aristas.

Dado que estos algoritmos tienen un costo computacional muy elevado, ya que se debe realizar una búsqueda intensiva de los centros y estructuras de las cajas convirtiéndose en un problema intratable computacionalmente. Se propone una heurística para estrategias evolutivas y recocido simulado, con el fin de reducir el número de computacionesque se deben realizar.

En el capitulo 1, se mostrará la definición del problema y objetivos de este proyecto. En el capitulo 2, se realiza una revisión de la literatura existente para el análisis de multifractalidad y robustez. En los capítulos 4 a 8, se muestra el proceso que se realizó para realizar el análisis de multifractalidad y robustez en algunas redes complejas seleccionadas. Finalmente en el capitulo 9, se presentan las conclusiones y los trabajos futuros.
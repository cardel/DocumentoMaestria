Las redes complejas permiten modelar sistemas observados, como es el caso de los sistemas biológicos\cite{Costa2008}, en los cuales se pueden realizar modelos de interacción entre proteínas, de organismos en un ecosistema, mecanismos de comunicación celular, entre otros. Otro ejemplo son los sistemas de transporte se modelan a partir de las interacciones entre los vehículos, los pasajeros y el sistema vial de una ciudad\cite{Wu2018}. Así mismo, los sistemas sociales se modelan a partir de las interacciones entre las personas, un caso de esto es la persuasión como un proceso en cascada\cite{Huang2016}. Otro campos de estudio son las redes que se generan a partir de iteraciones entre organizaciones\cite{6676578}, las redes de citaciones de artículos científicos\cite{Zhang2013}, entre otras. En consecuencia, las redes complejas son una herramienta que proporciona un soporte matemático para el estudio de diversos problemas en diferentes áreas del conocimiento.

En las ciencias de la computación las redes complejas\cite{BarabasiNetwork} son representadas como grafos, en los cuales los vértices o nodos representan una entidad en el modelo y las aristas las relaciones entre ellas. A partir de esta representación se pueden estudiar algunas propiedades estructurales de las redes, como es la distribución de grado que permite visualizar cómo están conectados los vértices en la red. Otras propiedades estructurales son las medidas de cercanía por centralidad y por intermediación que indican si un vértice hace parte de los caminos más cortos entre un par de vértices, así como la cercanía por vecindad, que se utiliza para medir que tan cerca está del centro de la red. Adicionalmente, una medida de gran importancia es la longitud de los caminos promedios entre un cada par de vértices, de esta forma diferentes medidas se propone para caracterizar elementos clave dentro de la topología de una red.

En la literatura, de acuerdo a las propiedades estructurales de las redes complejas existen varias categorizaciones, de las cuales las más utilizadas son: redes libres de escala y redes de mundo pequeño. Las redes libres de escala según Barabasi\cite{Albert2002} se caracterizan por tener una distribución de grado acorde a una ley de potencia, en la cual pocos vértices tienen un grado muy elevado (hubs) y un gran número de estos con un grado pequeño. Otra característica de las redes libres de escala es que el coeficiente de agrupación decrece a medida que el grado de los hubs se incrementa, esto implica que los diferentes componentes se encuentran conectados por los hubs y además estas redes presentan una gran robustez a fallas aleatorias. Por su parte, las redes de mundo pequeño según Amaral y otros\cite{Amaral2000} son aquellas que presentan las siguientes propiedades: primero, la distancia más corta promedio entre un par de vértices es proporcional al logaritmo natural del número de nodos y el coeficiente de agrupamiento es grande, lo que indica que los vértices de la red tienden a estar fuertemente conexos.

Sin embargo, dada la variedad de estructuras y formas de las redes complejas, estas medidas no son suficientes para estudiar la estructura de redes complejas debido a que las consideran como un todo\cite{Song2005}. El análisis multifractal aportar más información\cite{Li2014}, ya que permite estudiar la topología de las estructuras interna de una red. Por otro lado, la robustez\cite{Schneider2011}, permite analizar cómo cambia la estructura interna de las redes complejas a medida se comporta a perturbaciones de su estructura.

Para la medida de robustez, se tienen algoritmos basados en cobertura por cajas\cite{Song2007}, los cuales son utilizados en el procesamiento de imágenes para el análisis fractal, con la diferencia que en redes complejas se utiliza un sistema de referencia basado en el diámetro de la red. La idea, es observar que la dimensión fractal $L_B$ es la relación entre el diámetro de las cajas $R_B$ y número de cajas $N_B$, que está dado por $N_B=R_{B}^{L_B}0$. Sin embargo, esta medida ha demostrado no ser suficiente para el estudio de las estructuras internas de las redes\cite{Wang2012}, por ello se han desarrollado variantes que permiten realizar un análisis multifractal, el cual arroja más información sobre las estructuras de la red a medida que la red es escalada por un factor $q$. En el caso de la robustez, el análisis consiste en estudiar las propiedades estructurales de las redes a medida que esta pierde nodos o aristas, seleccionados a partir de alguna estrategia de ataque.

El estudio de una relación entre la multifractalidad y robustez, promete proveer un marco de trabajo para el estudio de las redes complejas, ya que, permite estudiar cómo se comporta la estructura de la red a medida que esta se ve afectada por algún tipo de ataque. La metodología de este estudio consiste en observar cómo se comportan las dimensiones fractales de una red a medida que se pierden nodos y aristas bajo alguna estrategia de eliminación. Esto permite ver cómo varia la estructura interna de acuerdo a al tipo de ataque que se somete a una red.

Dado que estos algoritmos tienen un costo computacional muy elevado y se debe realizar una búsqueda intensiva de los centros y estructuras de las cajas convirtiéndose en un problema intratable computacionalmente. Se propone una heurística para estrategias evolutivas y recocido simulado, con el fin de reducir el número de computaciones que se deben realizar.

Al realizar un conjunto de experimentos con diferentes redes complejas, se encuentra que la estructura de una red de mundo pequeño tiende a ser una sola o monofractal, ya que las curvas de dimensión fractal no presentan una variación significativa a diferencia de las redes libres de escala. Así mismo, en el caso de la relación entre multifractalidad y robustez, la estructura interna de las redes libres de escala no presenta variación ante ataques aleatorios a diferencia de ataques que afectan a los hubs; en las redes de mundo pequeño, se encuentra una variación insignificante en la dimensión fractal, ya que su estructura interna es compacta. A partir de estos dos resultados, se encuentra un campo de estudio para ampliar el significado de esta relación.

En el capitulo 1, se mostrará la definición del problema y objetivos de este proyecto. En el capitulo 2, se realiza una revisión de la literatura existente para el análisis de multifractalidad y robustez. En los capítulos 4 a 7, se muestra el proceso que se realizó para realizar el análisis de multifractalidad y robustez en algunas redes complejas seleccionadas. Finalmente, en el capitulo 8 se presentan las conclusiones y los trabajos futuros.
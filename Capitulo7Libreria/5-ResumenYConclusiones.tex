\section{Resumen y conclusiones del capitulo}

El uso de SNAP como librería para el manejo de redes complejas, que al utilizar una representación interna optimizada permite realizar búsquedas más eficientes. Además al estar escrita en C++, permite manejar redes de gran tamaño, cuyo único limitante es la memoria del equipo.

Sin embargo, los algoritmos de análisis multifractal requieren una gran capacidad de memoria y cómputo, por lo que para redes grandes se debe estudiar la posibilidad de trabajar en equipos con gran cantidad memoria y procesadores. Así mismo, en la posibilidad de distribuir sus tareas.

EL API provisto por la librería Sphinx, permite a un usuario de librería explorar las diferentes funciones que se proveen y realizar búsqueda. Así mismo, en el API hay ejemplos de uso de la librería.
\section{Resumen y conclusiones del capitulo}

La selección de SNAP se debe a que utiliza una representación interna de matrices dispersas, la cual permite realizar operaciones sobre las redes con menor costo computacional que otras opciones disponibles. Además al estar escrita en C++, elimina las limitaciones que tienen las librerías que requieren ser gestionadas por un interpretador.

Sin embargo, los algoritmos de análisis multifractal requieren una gran capacidad de memoria y cómputo, por lo que para redes grandes se debe estudiar la posibilidad de trabajar en equipos con gran cantidad memoria y procesadores. Así mismo, en la posibilidad de distribuir sus tareas.

La documentación provista por la librería Sphinx permite a un usuario de librería explorar las diferentes funciones que se proveen y realizar búsqueda. Así mismo, la flexibilidad del generador de documentación permite ejemplos de uso de la librería.
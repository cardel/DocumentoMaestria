\section{Funciones provistas}

\subsection{Análisis multifractal}

Las funciones provistas para el análisis multifractal son:

\begin{lstlisting}[language=python]
FSBCAlgorithm.FSBCAlgorithm.FSBCAlgorithm(g, minq, maxq, percentNodesT, centerNodes=array([], dtype=float64))

BCAlgorithm.BCAlgorithm.BCAlgorithm(g, minq, maxq, percentNodesT, centerNodes=array([], dtype=float64))

SBAlgorithm.SBAlgorithm.SBAlgorithm(g, minq, maxq, percentSandBox, repetitions, centerNodes=array([], dtype=float64))
\end{lstlisting}

Estas funciones permiten ingresar la configuración de los diferentes algoritmos.

\begin{itemize}
    \item maxq y minq, son los valores de q. El rango debe tener 0 incluido
    \item g es el grafo en formato SNAP
    \item porcentNodesT, percentSandBoxes, son el número de conjuntos que serán seleccionados como centros
    \item repetitions, para SandBox, al ser un algoritmo de elección aleatoria de centros debe repetirse y tomar promedios de los resultados
    \item centernodes, utilizados por las estrategias de Inteligencia Artificial que proveen al algoritmo centros ya calculados, por defecto es vacío
\end{itemize}

\subsection{Análisis de robustez}

\begin{lstlisting}[language=python]
robustness.robustness.robustness_analysis(graph, typeRemoval, minq, maxq, percentSandBox, repetitions, temperature=0, sizePopulation=0, iterationsGenetic=0, percentCrossOver=0, percentMutation=0, degreeOfBoring=0)
\end{lstlisting}

\begin{itemize}
    \item maxq y minq, son los valores de q. El rango debe tener 0 incluido
    \item typeRemoval, es el tipo de ataque, las opciones son:
    \begin{enumerate}
        \item Random: para aleatorio
        \item Degree: para grado
        \item Centrality: para centralidad
        \item Genetic: para genético
        \item Simulated: para recocido simulado
    \end{enumerate}
    \item percentSandBox, parámetro del algoritmo SandBox
    \item repetitions, parámetro del algoritmo SanBox
    \item temperature, parámetro del algoritmo de recocido simulado
    \item  sizePopulation, iterationsGenetic, percentCrossOver, percentMutation, degreeOfBoring. Parámetros del algoritmo evolutivo
\end{itemize}

\subsection{Algoritmos evolutivos}

\begin{lstlisting}[language=python]
Genetic.Genetic.Genetic(g, minq, maxq, sizePopulation, iterations, percentCrossOver, percentMutation, degreeOfBoring, typeAlgorithm)
\end{lstlisting}

\begin{itemize}
    \item percentCorssOver: Porcentaje de cruce
    \item percentMutacion: Porcentaje de mutación
    \item degreeOfBoring: Número de generaciones en las cuales se detiene el algoritmo si el mejor individuo no mejora
    \item typeAlgorithm. Tipo de algoritmo que se aplica el análisis multifractal, puede ser:
    \begin{enumerate}
        \item SB para SandBox
        \item BC Para Box Compact Counting
        \item FSBC para Fixed Size Box Counting
    \end{enumerate}
\end{itemize}

\subsection{Algoritmos de recocido simulado}

\begin{lstlisting}[language=python]
SimulatedAnnealing.SimulatedAnnealing.SA(g, minq, maxq, percentNodes, sizePopulation, Kmax, typeAlgorithm)
\end{lstlisting}

Los parámetros de este algoritmo han sido explicados anteriormente. El funcionamiento es simular al evolutivo, ya que se calculan los centros de las cajas y estas son pasadas a alguno de los algoritmos de análisis multifractal.
\section{Glosario}
\begin{itemize}

\item{\textbf{Algoritmo genético} \cite{WGOEB} Los algoritmos genéticos son estrategias adaptativas para la solución de diversos problemas de búsqueda y de optimización usando conceptos de evolución. Un algoritmo genético está definido por:
	\begin{enumerate}
		\item Una población inicial que se genera de manera aleatoria, cada individuo es una solución válida a un problema que se busca solucionar.
		\item Un método de selección que selecciona los mejores individuos de una población.
		\item Cruce, que es tomar dos individuos que han sido seleccionados y combinarlos para generar más individuos que recojan sus características.
		\item Mutación, que es alterar características a unos pocos individuos generados en el cruce.
		\item Los pasos 2,3, y 4 se repiten un número determinado de veces o cuando los mejores individuos cumplan ciertos requisitos definidos por el programador.
	\end{enumerate}
La idea principal es que el algoritmo a medida que itera se va acercando a la solución óptima del problema que se intenta solucionar.
}

\item \textbf{Ancho de banda:} \cite{Cuadro} En comunicaciones se define como el rango de frecuencias necesario para transmitir una señal. 

\item \textbf{ANE:} \cite{ANE} Agencia Nacional del Espectro, es una entidad del gobierno colombiano, adscrita al Ministerio Tecnologías de la información y las comunicaciones (MinTIC), encargado de gestionar el uso del espectro radioeléctrico, excepto para la televisión.

\item \textbf{Banda de frecuencias:} \cite{Cuadro}  {División del espectro radioeléctrico que define un conjunto de ondas electromagnéticas cuyas frecuencias se encuentran dentro de un límite inferior y un límite superior indicados explícitamente.
Se definen nueve grandes bandas de frecuencias: VLF, LF, MF, HF, VHF, UHF, SHF, EHF y la banda de frecuencias que comprende frecuencias superiores a 300 GHz. Estas a su vez están subdividas en otras bandas más pequeñas a las cuales se atribuyen los distintos servicios de
radiocomunicación.
\\
Las diferentes bandas de frecuencia están especificadas así:
\begin{table}[H]
	\centering
	\label{tabla:bandas}
	\caption{Especificación de bandas de frecuencia.}
	\caption*{Tomado del cuadro nacional de atribución de frecuencias, página 47.}
	\begin{tabular}{|p{5.5cm}|p{5.5cm}|}
		\hline
			\cellcolor[gray]{0.9} \textbf{Banda} & \cellcolor[gray]{0.9}\textbf{Frecuencias} \\
		\hline
		Very Low Frecuency VLF & 3 - 30 kHz \\
		\hline
		Low Frecuency LF & 30 - 300 kHz\\
		\hline
		Médium Frecuency MF & 300 - 3000 kHz\\
		\hline
		High Frecuency HF & 3 - 30 MHz\\
		\hline
		Very High Frecuency	VHF & 30-300 MHz\\
		\hline
 		Ultra High Frecuency UHF & 300-3000 MHz\\
 		\hline
		Super High Frecuency SHF & 3 - 30 GHz\\
		\hline
		Extremely High Frecuency EHF & 30-300 GHz\\
		\hline
	\end{tabular}
\end{table}

}

\item \textbf{Banda de frecuencias específica o rango de frecuencia:} \cite{Cuadro} Es una división específica del espectro en donde se presta uno o más servicios de radiocomunicación.

\item \textbf{Canal de comunicación:} \cite{Cuadro} Son los segmentos en que se divide una banda de frecuencia. En el proceso de gestión del espectro son los elementos que son asignados a los operadores.

\item \textbf{Espectro electromagnético:} \cite{Cuadro} Es la división energética de las ondas. Las bandas del espectro electromagnético son: radiofrecuencia, microondas, infrarrojo, espectro visible, ultravioleta, rayos X y rayos gamma.

\item \textbf{Espectro radioeléctrico:} \cite{Cuadro} Conjunto de frecuencias electromagnéticas entre 3Hz y 3GHz. El espectro radioeléctrico es un subconjunto menos energético del espectro electromagnético que se utiliza para la transmisión de señales de radio, televisión, celular y otras.

\item \textbf{Estación:} \cite{Cuadro}: Uno o más transmisores o receptores, o una combinación de transmisores y receptores, incluyendo las instalaciones accesorias, necesarios para asegurar la prestación de un servicio en un lugar determinado. Existen estaciones fijas las cuales permanecen en un lugar determinado y móviles que se encuentran en movimiento dentro de una zona determinada, como por ejemplo usuarios de telefonía celular.

\item \textbf{Frecuencia:} \cite{Cuadro} Es el numero de repeticiones de una onda durante un segundo, se mide en Hertz (Hz) que indica ciclos por segundo.

\item \textbf{ITU:} \cite{ITU} En inglés  - \textit{International Telecommunication Union} -, en español - Unión Internacional de Telecomunicaciones -. Es un organismo de las Naciones Unidas, encargado de regular las telecomunicaciones a nivel mundial; genera normativas conocidas como recomendaciones, que no son de obligatorio cumplimiento para diferentes áreas en las telecomunicaciones.

\item \textbf{Longitud de onda:} \cite{Cuadro} Es el período espacial de las ondas, en otras palabras la distancia en que se repiten.

\item \textbf{Operador:} Empresa privada o pública que presta servicios en el espectro radioeléctrico.

\item \textbf{Periodo de señal:} \cite{Cuadro} Es el tiempo que requiere una señal para repetirse, en otras palabras es el inverso de la frecuencia.

\item \textbf{Potencia de señal:} \cite{Cuadro} Es la energía de transmisión de una señal por unidad de tiempo, se mide en Wattios (W).

\item \textbf{Interferencia de señal:} \cite{Cuadro} Es la afectación en el nivel de potencia y periodo de una señal que emite en una frecuencias, producida por otras señales transmitidas en frecuencias cercanas.

\item \textbf{Modelo de satisfacción  de restricciones:} \cite{GCP} Es un conjunto de variables, dominios finitos y ecuaciones, que permiten solucionar un problema usando programación por restricciones.

\item \textbf{MinTIC:} \cite{MinTIC} Acrónimo de Ministerio de Tecnologías de la Información y las Comunicaciones, es uno de los ministerios del estado colombiano, su función es regular y promover el uso de las tecnologías de comunicación en todo el país.

\item \textbf{Servicio:} Actividad que busca satisfacer las necesidades de un cliente, para éste proyecto de grado, está enfocado a los que utilizan el espectro radioeléctrico.

\item \textbf{Sistemas de gestión de contenidos Web:} \cite{Gestor} Es una herramienta que permite crear, editar, modificar y publicar contenidos Web.

\item \textbf{Telecomunicación:} \cite{Cuadro}: Toda transmisión, emisión o recepción de signos, señales, escritos, imágenes, sonidos o informaciones de cualquier naturaleza por hilo, radioelectricidad, medios ópticos u otros sistemas electromagnéticos.

\item \textbf{XML}: \cite{XML} Es un lenguaje basado en etiquetas, muy flexible que permite estandarizar la representación de datos.

\end{itemize}

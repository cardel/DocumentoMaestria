En este proyecto de grado se busca obtener una propuesta de solución al problema de gestión del espectro radioeléctrico usando programación con restricciones, con una aplicación prototipo en la que se puedan realizar desarrollos a futuro, por esta razón no se busca obtener un producto final.
\\ \\
Por lo tanto se han aplicado algunas limitaciones al desarrollo de la aplicación, para permitir su concepción, diseño e implementación dentro del contexto y alcance de un trabajo de grado de pregrado.

\section{Alcances metodológicos}	

La gestión del espectro se refiere a la asignación, actualización y remoción de la asignación de frecuencias para los operadores en el espectro radioeléctrico.  En la aplicación prototipo no se cambiarán ni eliminarán manualmente las bandas que han sido asignadas a partir de una solución que se ha obtenido automáticamente en el aplicativo.
\\ \\
No se va trabajar el estudio de interferencias; se da por descontado una distancia de $sep$ canales entre dos operadores dentro de una banda para garantizar no se interfieran entre sí.
\\ \\
La asignación es jerárquica en orden territorial por ejemplo, un canal asignado a un operador a nivel nacional está también asignado en cada región, departamento y municipio del país.
\\ \\
La asignación por entes territoriales no considera en detalle la asignación en sus divisiones por ejemplo, para el caso de un departamento no se toma la asignación en detalles de sus municipios, sino qué canales se encuentran ocupados en uno o más municipios. Por lo tanto una solución óptima en un ente territorial no lo es necesariamente en sus divisiones. Esta decisión se tomó por el gran tamaño que tomaban las entradas para describir el detalle de la asignación en cada división de un ente territorial.
\\ \\
En la práctica se establecen topes de espectro para cada operador en una banda, éste tope está definido en Hz, para efectos de simplificar, se toma un número de canales por banda, es de anotar que cada canal tiene un tamaño en Hz establecido por su frecuencia inicial y final.
\section{Alcances prácticos}

Con respecto al proceso de asignación se toman las siguientes consideraciones en el diseño de la aplicación:

\begin{itemize}
	\item El acceso a los datos utilizados en la práctica de la división territorial y asignación de frecuencias es restringido, por lo que en el proyecto se utilizan datos estimados.
	\item El modelo de los canales es variable en el cuadro nacional de atribución de frecuencias, en este trabajo de grado solamente se supone un modelo con una frecuencia de transmisión $T_{x}$ y una de recepción $R_{x}$.
	\item Se trabaja entre las frecuencias entre 26 MHz y 60G GHz, al omitir las frecuencias de servicio móvil marítimo y servicio fijo marítimo y las frecuencias más altas que no se encuentran canalizadas.
	\item Solamente se trabaja con las bandas de frecuencias definidas por servicios, las cuales se encuentran canalizadas, se ignora la información sobre la estructura de las bandas definidas por la ITU. Estas bandas se les denomina en el proyecto y aplicativo como rangos de frecuencia.
	\item La zonas territoriales del Colombia se dividen en entes territoriales, departamentos y municipios.
	\item La base de datos sólo contiene la información relevante para el proyecto, los datos de rangos de frecuencia y servicios han sido extraídos del cuadro nacional de atribución de frecuencias.
\end{itemize}


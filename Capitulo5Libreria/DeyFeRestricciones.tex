\section{Debilidad y flexibilidad de restricciones}

Existen algunas instancias del problema que no cumplen una o más restricciones, por lo tanto, se diseñan estrategias de flexibilidad y debilidad de restricciones para realizar su procesamiento.

\subsection{Restricciones flexibles}

La flexibilidad en las restricciones consiste en que algunas entradas del problema toman valores que deshabilitan el efecto de una o más restricciones del problema.

\subsubsection{No considerar tope}

En este caso $Tope=C$, se fuerza a que el tope máximo de cada operador sea el número de canales de la banda.

\subsubsection{No considerar separación entre canales}

En este caso $Sep=0$, es decir que la separación entre canales de operadores diferentes es cero.

\subsubsection{Frecuencias asignadas a operadores con requerimientos se pueden mover}

Se toma $CAC = 0$, lo que evita que se conserva la asignación de los operadores que solicitan canales.

\subsubsection{Más de un operador por canal}

Se cumple $R>1$, en este caso también $Sep=0$, ya que no tiene sentido considerar separación en un caso donde varios operadores pueden compartir un canal.

\subsection{Debilidad de restricciones}

La estrategia utilizada en el proyecto para manejar fortaleza y debilidad de las restricciones, consiste en permitir al usuario elegir que restricciones son flexibles, en seis posibles iteraciones.
\\\\
Cada iteración es independiente de las otras. El usuario define qué restricciones flexibiliza en cada una de ellas y a partir de los resultados de cada una elige cuál tomar como solución final. 
\\\\
Las iteraciones en el aplicativo se ejecutan al mismo tiempo debido a que son independientes entre sí y los resultados de cada una son desplegados al usuario.

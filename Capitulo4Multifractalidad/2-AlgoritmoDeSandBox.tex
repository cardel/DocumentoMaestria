\section{Algoritmo SandBox}
\label{cap4:seccionSB}

Liu y otros\cite{Liu2015} han realizado un estudio de los diferentes algoritmos de Boxcounting encontrando que estos requieren un gran tiempo de cómputo, ya que se realiza la búsqueda de una configuración óptima de centros para obtener una estructura de cubrimiento, en la cual cada caja tiene el mismo número de nodos o se obtiene el número mínimo de cajas necesario para el cubrimiento de la red.

Su propuesta conocida como algoritmo de caja de arena o SandBox, consiste en escoger un número de centros de cajas aleatorios y permitir que las cajas se solapen. Dado que se permite solapamientos, como medida se utilizará el promedio del número de nodos en cada caja y con este valor se realiza le medición.

El proceso de este algoritmo es el siguiente:

\fbox{\begin{minipage}{16.5cm}
 \begin{itemize}
     \item Inicialmente, seleccione un porcentaje nodos como centros de las cajas.
     \item Seleccione el radio $r \in [1,d]$ de cada caja de arena que será usado para cubrir la red.
     \item Cuente el número de nodos en cada caja de arena de radio $r$, denote este número como $M(r)$.
     \item Para todos los centros elegidos calcule el promedio estadístico de $\overline{M(r)^(q-1)}$
 \end{itemize}
 \end{minipage}}

 La regresión lineal se realiza entre $\ln(\overline{M(r)^(q-1)})$ y $\ln(\frac{r}{d})$
 
Debido a que éste método elige centros aleatorios, se recomienda repetirlo varias veces para mejorar la precisión.
 

 
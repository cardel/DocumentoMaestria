\section{Estudio de los algoritmos de inteligencia artificial}

Los algoritmos Genético y de Recocido simulado son comparados aplicando pruebas en las siguientes redes

\begin{itemize}
    \item Red A: Libre de escala 2000 nodos
    \item Red B: Mundo pequeño 5000 nodos, probabilidad de reconexión 5\%
    \item Red C: Red aleatoria 3733 nodos
    \item Red D: Red real del hongo cerevisiae
\end{itemize}

En el caso del algoritmo genético también es mostrada la evolución de la función de evaluación.

\section{Comparativa de tiempos de ejecución}

En la tabla \ref{tab:executionTimes} son analizados los tiempos de ejecución de cada uno de los algoritmos, de acuerdo al protocolo de pruebas especificado en el Anexo B de la página \pageref{AnexoB}

\begin{table}[H]
    \centering
    \begin{tabular}{|p{1cm}|p{1cm}|p{1.1cm}|p{1cm}|p{1.5cm}|p{1.5cm}|p{1.5cm}|p{1.5cm}|p{1.5cm}|p{1.5cm}|}
         \hline
         \textbf{Red} & \textbf{FSBC} & \textbf{GCC} & \textbf{SB} & \textbf{Genético FSBC} & \textbf{Recocido FSBC} &
         \textbf{Genético BCC} & \textbf{Recocido BCC} &
         \textbf{Genético SB} & \textbf{Recocido SB} \\
         \hline
         Red A & 10788 & 10715 & 1193 & 7097 & 5092 & 6324 & 4995 &4644 & 6653 \\
         \hline
    \end{tabular}
    \caption{Tiempos de ejecución para diferentes redes en segundos}
    \label{tab:executionTimes}
\end{table}
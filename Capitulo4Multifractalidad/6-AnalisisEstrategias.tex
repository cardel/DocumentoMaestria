\section{Estudio de los algoritmos de inteligencia artificial}

\label{cap4:analisisIA}

Los algoritmos Genético y de Recocido simulado son comparados aplicando pruebas en las siguientes redes

\begin{itemize}
    \item Red A: Libre de escala 2000 nodos
    \item Red B: Mundo pequeño 5000 nodos, probabilidad de reconexión 5\%
    \item Red C: Red aleatoria 3733 nodos
    \item Red D: Red real del hongo cerevisiae
\end{itemize}

En el caso del algoritmo genético también es mostrada la evolución de la función de evaluación.


\subsection{Red libre de escala}

\begin{figure}[H]
    \centering
    \includegraphics[scale=0.7]{{Capitulo4Multifractalidad/imagenesIA/grafica_Dq20180502_203759ScaleFree2000Nodes.txt}.png}
    \caption{Multifractalidad en red libre de escala con 2000 nodos}
\end{figure}

\begin{figure}[H]
    \centering
    \includegraphics[scale=0.7]{{Capitulo4Multifractalidad/imagenesIA/grafica_Fitness20180502_203759ScaleFree2000Nodes.txt}.png}
    \caption{Evolución de la función de evaluación en red libre de escala con 2000 nodos}
\end{figure}

\subsection{Red de mundo pequeño}

\begin{figure}[H]
    \centering
    \includegraphics[scale=0.7]{{Capitulo4Multifractalidad/imagenesIA/grafica_Dq20180506_141058SmallWorld5000NodesRewire01.txt}.png}
    \caption{Multifractalidad en red de mundo pequeño de 5000 nodos con p = 10\%}
\end{figure}

\begin{figure}[H]
    \centering
    \includegraphics[scale=0.7]{{Capitulo4Multifractalidad/imagenesIA/grafica_Fitness20180506_141058SmallWorld5000NodesRewire01.txt}.png}
    \caption{Evolución de la función de evaluación en red de mundo pequeño de 5000 nodos con p = 10\%}
\end{figure}

\subsection{Red aleatoria}

\begin{figure}[H]
    \centering
    \includegraphics[scale=0.7]{{Capitulo4Multifractalidad/imagenesIA/grafica_Dq20180508_231031Random3373Nodes5978.txt}.png}
    \caption{Multifractalidad en red aleatoria con 3373 nodos}
\end{figure}

\begin{figure}[H]
    \centering
    \includegraphics[scale=0.7]{{Capitulo4Multifractalidad/imagenesIA/grafica_Fitness20180508_231031Random3373Nodes5978.txt}.png}
    \caption{Evolución de la función de evaluación en red aleatoria con 3373 nodos}
\end{figure}

\subsection{Red cerevisae}

\begin{figure}[H]
    \centering
    \includegraphics[scale=0.7]{{Capitulo4Multifractalidad/imagenesIA/grafica_Dq20180508_182332cerevisiae.txt}.png}
    \caption{Multifractalidad en red real de hongo cerevisiae}
\end{figure}

\begin{figure}[H]
    \centering
    \includegraphics[scale=0.7]{{Capitulo4Multifractalidad/imagenesIA/grafica_Fitness20180508_182332cerevisiae.txt}.png}
    \caption{Evolución de la función de evaluación en red real de hongo cerevisiae}
\end{figure}

\subsection{Análisis de los algoritmos}

Se puede observar que:

\begin{enumerate}
    \item No es sencillo estimar un número de iteraciones para los algoritmos de inteligencia artificial, en algunos casos el resultados es muy cercano a los deterministas, en otros no
    \item Dado que el tiempo de ejecución en cualquier caso es muy amplio, se requiere un soporte teórico para conocer el número de iteraciones óptimo para cada estrategia
    \item Las gráficas de la evolución de la función de evaluación muestran que la heurística usada debe ser mejorada para validar cuales son los mejores centros
\end{enumerate}

\section{Comparativa de tiempos de ejecución}

En la tabla \ref{tab:executionTimes} son analizados los tiempos de ejecución de cada uno de los algoritmos, de acuerdo al protocolo de pruebas especificado en el Anexo B de la página \pageref{AnexoB}

\begin{table}[H]
    \centering
    \begin{tabular}{|p{1cm}|p{1cm}|p{1.1cm}|p{1cm}|p{1.5cm}|p{1.5cm}|p{1.5cm}|p{1.5cm}|p{1.5cm}|p{1.5cm}|}
         \hline
         \textbf{Red} & \textbf{FSBC} & \textbf{BCC} & \textbf{SB} & \textbf{Genético FSBC} & \textbf{Recocido FSBC} &
         \textbf{Genético BCC} & \textbf{Recocido BCC} &
         \textbf{Genético SB} & \textbf{Recocido SB} \\
         \hline
         Red A & 10788 & 10715 & 1193 & 7097 & 5092 & 6324 & 4995 &4644 & 6653 \\
         \hline
         Red B & 84833 & 85043  & 16598  & 48511  & 38744  &  35022 & 48511 & 35067 & 48360  \\
         \hline
         Red C & 271062 & 380427 & 19535  &  16508 & 20410  &  27972 & 16508 & 24480 & 15692  \\
         \hline
         Red D & 106396 & 106751 &  60063 & 75580  & 61520  & 94662  & 57915  & 68997 &  60096 \\
         \hline
    \end{tabular}
    \caption{Tiempos de ejecución para diferentes redes en segundos}
    \label{tab:executionTimes}
\end{table}

Se observa que en general los algoritmos FSBC y BCC tienen una mejora considerable cuando se les aplica estrategias de inteligencia artificial. Con respecto al algoritmo SB, se requiere realizar un estudio más amplio para llegar a una conclusión debido a que:

\begin{itemize}
    \item El número de iteraciones de SB depende del número de nodos
    \item El número de iteraciones de los algoritmos de IA depende los parámetros suministrados por el usuario
\end{itemize}

Lamentablemente, los algoritmos requiere gran capacidad de cómputo y tiempo de ejecución, por lo que esto se debe realizar en un estudio posterior.
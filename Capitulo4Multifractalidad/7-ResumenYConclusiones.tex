\newpage
\section{Resumen y conclusiones del capítulo}

En este capitulo, inicialmente se analizan 3 algoritmos presentes en la literatura: Box Counting Fixed Size (BCFS), Box Compact Counting (BCC) y SandBox (SB). Se encuentra que estos métodos se deben ejecutar un número no determinado de veces para hallar las dimensiones fractales, ya que dentro de sus pasos hay una selección aleatoria de centros de cajas.

Ninguno de los algoritmos estudiados garantiza un valor exacto de la dimensión fractal, por lo que se requiere realizar una búsqueda intensiva de los mejores centros para realizar el computo de las dimensiones fractales, en consecuencia, no son prácticos para redes con una gran cantidad de nodos. De hecho, en los artículos que son propuestos, el análisis es realizado con un conjunto selecto de redes de no más de 10000 nodos. Los algoritmos sólo proveen información de que tan heterogénea es la estructura interna de la red.

Estos algoritmos requieren un gran tiempo de cómputo, el cual depende del número de nodos, aristas y cómo esta conectada la red, por lo que proveer una medida de complejidad en términos de vértices o aristas no es práctica. En consecuencia, este estudio se realiza comparando el tiempo de cómputo en distintos ejemplos.

Como estrategia para reducir el tiempo de cómputo, se propone un algoritmo evolutivo y uno de recocido simulado, la idea es mediante un proceso distinto al de cubrimiento de cajas que debe estudiar toda la red elegir los centros de las cajas. La estrategia se basa en una heurística que consiste en buscar una configuración de nodos, los cuales tengan el mayor grado posible y se encuentren lo más alejados entre sí que sea posible.

Los resultados de la tabla \ref{tab:executionTimes}, dan indicios que las estrategias de inteligencia artificial mejoran el tiempo de ejecución de los algoritmos BCFS y BCC: Para el algoritmo SB, debido a que depende de una selección aleatoria y del número de repeticiones que se le indiquen, no se puede concluir si las estrategias propuestas lo mejoran o no. 
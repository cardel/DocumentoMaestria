En éste capítulo se abordan los conceptos necesarios para el entendimiento del problema de la gestión del espectro y las estrategias de solución utilizadas en este proyecto de grado.
\\\\
Es recomendado que el lector tenga formación básica en:

\begin{itemize}
	\item {Conceptos de física fundamental:
		\begin{itemize}
			\item Conocimiento básico acerca de física de ondas.
			\item Entendimiento básico acerca del espectro electromagnético.
			\item Efectos de la distancia, clima y otros que afectan la propagación de señales en el espacio.
		\end{itemize}
	}
	\item Conocimiento básico sobre tecnologías de comunicación inalámbrica.
	\item Conocer qué son problemas combinatorios y la dificultad de solucionarlos.
	\item Claridad en el concepto de modelo matemático.
	\item Información acerca de la diferencia entre aplicaciones de escritorio y Web.
\end{itemize}

A continuación se ilustran los términos más relevantes para el proyecto de grado, luego se hace una introducción a la gestión del espectro y finalmente se explican algunos conceptos sobre la programación por restricciones.

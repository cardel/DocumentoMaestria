\section{Algoritmos de inteligencia artificial en redes complejas}


\subsection{Patrones en redes complejas}

En las redes complejas se pueden identificar patrones estructurales, los cuales pueden ser
utilizados para su estudio. Li y otros\cite{Li2012IA} realizan un evaluación de las redes complejas de gran escala utilizando herramientas de minería de datos y reconocimiento de patrones, a través de una conversión de las redes en estructuras de nubes de datos, que facilita el proceso de descubrimiento de patrones en las redes. Como resultado de este trabajo se encuentra que si se proyectan las redes en un espacio multidimensional de parámetros se pueden encontrar patrones.

Otro aporte relevante en fue la conferencia \textit{Workshop on Structural, Syntactic, and Statistical Pattern Recognition}\cite{Jain2000}, realizada
en el año 2010, donde se presentaron trabajos sobre la descripción estructural de redes complejas y el uso de estrategias de inteligencia artificial para su estudio. Un ejemplo de ello son el uso de redes Markovianas para el estudio de redes complejas enmarcados en el estudio de la categorización de material multimedia en el estándar MPEG-7. Otro campo de estudio de esta conferencia fue el estudio del aprendizaje estructural, con el uso de ténicas como los vectores de cuantización (LVQ) para la descripción estructura del grafos, los resultados indican que se
pueden utilizar para la descripción de grafos.


Desde el enfoque del estudio de sistemas biológicos, Wuchty y otros\cite{Wuchty2005} realizan un estudio de la aplicación de diferentes técnicas a sistemas complejos que representan interacciones entre proteínas. Se busca integrar las redes complejas, el análisis de imágenes, el procesamiento de señales y la inteligencia artificial, mediante el estudio de aplicación de estas técnicas a varios casos de estudio, uno de ellos, consiste en estudiar esencialidad de las proteínas, que consiste en construir una red compleja a partir de sus interacciones. El análisis de este caso indica que existen propiedades como
la centralidad de nodos y propiedades de \textit{k-core} en la red, las cuales permiten describir la topología de la red.

\subsection{Algoritmos de inteligencia artificial aplicados en redes complejas}

Los trabajos donde se han utilizado técnicas de inteligencia artificial para la medición de parámetros en redes complejas están basados principalmente en estrategias para optimizar algoritmos ya existentes, como es el caso del trabajo de Zhou y otros\cite{Zhou2007}, en donde se explica la autosimilaridad en redes complejas celulares mediante un método de cobertura de aristas utilizando algoritmos de recocido simulado aplicado dentro del algoritmo cobertura de cajas. Esta técnica consiste en encontrar un número mínimo de $N$ cajas de tamaño $l$ que pueda cubrir las aristas de la red. Las pruebas se realizaron sobre 43 redes celulares, encontrando que estas redes presentan una distribución de ley de potencia, donde la dimensión fractal es $[2,67 \pm 0,15]$. Sin embargo, este resultado no puede concluir que exista una distribución de grado de ley de potencia en todas las redes celulares.

Para los algoritmos evolutivos, un elemento de la literatura clave es libro \textit{Multiobjective Evolutionary Algorithms on Complex Networks} de  Michell Kirley y Robert Stewart\cite{Michel2007}, en donde se realiza una introducción del uso
de algoritmos evolutivos multiobjetivo en redes complejas. Este libro ofrece diferentes acercamientos teóricos para permitir mapear los nodos de una red dentro de los individuos de una población, crear reglas de selección y evolución de tal manera se pueda modelar una red compleja y así aplicar algoritmos para la medición de parámetros en ellas.

En la aplicación técnicas de inteligencia artificial en problemas de redes complejas es el realizado por Chien\cite{Chien1998}, en el cual se plantea el uso de técnicas de planificación basadas en inteligencia
artificial, en redes donde se representan las interacciones entre diferentes componentes de software. Se utiliza un algoritmo conocido como \textit{Multimission VICAR Planner}(MVP) utilizado en el procesamiento digital de imágenes, este algoritmo permite a partir de un conjunto
de imágenes de interés y la especificación de un estado deseado, realizar un conjunto de pasos para alcanzar dicho estado; en este algoritmo se construye un grafo que representa la estructura de almacenamiento de las imágenes. La aplicación MVP al problema de componentes de software permite estructurarlos considerando sus dependencias
y categorías. Los resultados muestran una mejoría en el proceso de configuración y reconfiguración de estos módulos. Otro trabajo relevante, es el realizado por Jian Liua y Tingzhan Liub\cite{Liu2010}, dónde se utilizan una combinación entre el algoritmo de recocido simulado y el algoritmo K-Means para encontrar comunidades dentro de redes complejas. Las comunidades se pueden ver cómo grupos de nodos que comparten características similares representadas por el peso de las conexiones dentro de la
red. En este trabajo, se encuentra que el uso de estas técnicas combinadas ayudan a encontrar más rápidamente las comunidades dentro de la red debido a que mejora la elección del centro de cada clúster en el procesos del algoritmo K-Means, debido al proceso de calentamiento o enfriamiento que provee el algoritmo de recocido simulado, y además, se pueden encontrar comunidades sin las necesidad de conocimiento previo acerca de la estructura de las comunidades.

En el diseño heurísticas para aplicar algoritmos de inteligencia artificial en redes complejas, en la literatura se encuentran varios trabajos relevantes, uno de ellos es realizado por Chen y otros\cite{Chen2009}, el cual se propone una heurística para la detección de comunidades en redes complejas utilizando como base la existencia de relaciones cercanas entre los nodos de una misma comunidad, para calcular el centro de cada comunidad en base las relaciones entre nodos y no directamente al promedio de los parámetros de cada nodo. Los resultados de esté articulo demuestran a-priori que el uso de esta estrategia sobre cuatro redes de prueba, es más eficiente para la detección de comunidades en redes complejas que otras estrategias como es el caso del algoritmo K-Means, debido a que los centros de cada comunidad convergen más rápidamente.
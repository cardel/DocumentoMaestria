\section{Estado del arte de análisis fractal y multifractal de redes complejas}

\subsection{Análisis multifractal}

Muchos estudios han demostrado que las redes complejas permiten modelar y caracterizar la dinámica de sistemas complejas presentes en la vida real\cite{BarabasiNetwork}. Los nodos de una red representan los elementos o agentes y las aristas representan la relación entre estos. Las aristas tienen un significado dentro del sistema y su existencia se debe a una relación o una regla que rige los agentes. Las redes complejas pueden ser estudiadas a partir de sus propiedades estructurales, como lo son la distribución de grado, las medidas de centralidad, su diámetro, la existencia de comunidades dentro de la red, entre otras. Recientemente, las redes complejas han sido estudiadas desde sus propiedades geométricas, una de estas medidas es la fractalidad o autosimilaridad, la cual permite estudiar cómo es la estructura interna de una red compleja\cite{Estrada2011}.

Para la medida de fractalidad, Song y otros\cite{Song2005} han propuesto un algoritmo generalizado de conteo de cajas. Este algoritmo es ampliamente utilizado en espacios geométricos, los cuales son transformados en espacios de acuerdo a la distancia más corta entre cada par de nodos, para así ser aplicado en redes complejas. Ellos encontraron que muchas redes complejas son auto-similares bajo ciertos valores de escala. En un trabajo posterior de los mismos autores\cite{Song2007}, se proponen una serie de algoritmos para calcular la dimensión fractal, entre los cuales se tiene, una aproximación al problema de coloreo de grafos y un algoritmo de conteo basado en el solapamiento de las cajas.

Posteriormente, Kim y otros\cite{Kim2007A}\cite{Kim2007B} proponen un algoritmo para la dimensión fractal en redes libres de escala basada en su esqueleto, el cual es obtenido a partir del árbol generador mínimo. Esta aproximación permite determinar la dimensión fractal más rápidamente que las anteriores estrategias, debido a que no se requiere explorar toda la red. A partir de esta idea, Zhou y otros\cite{Zhou2007} diseñan un algoritmo basado en el problema de cobertura de aristas, el cual es aplicado en redes complejas biológicas. 

Según Lee y Jung\cite{Lee2006} la dimensión fractal no es suficiente para describir las propiedades de autosimilaridad en redes complejas, se propone como herramienta el análisis multifractal, el cual además de considerar la dimensión fractal toma en cuenta la escalabilidad. Se encuentra preliminarmente que el análisis multifractal permite describir de mejor forma la distribución del coeficiente de agrupamiento en una red compleja.

Para el análisis multifractal se han propuesto algunos algoritmos para calcular los exponentes de masa\cite{Liu2014}, los cuales describen los cambios de densidad de la red compleja a medida que esta es escalada. Basado en los algoritmos de conteo de cajas para análisis fractal, se ha introducido una variante denominada algoritmo de conteo de cajas compacto\cite{Furuya2011}. Esta variante, toma como exponentes de masa el promedio del número de nodos elevado a un factor que representa la escala y calcula la dimensión fractal a partir de una regresión lineal entre el logaritmo de los exponentes de masa y el logaritmo del tamaño de las cajas, sin embargo, debido al gran número de computaciones que se deben realizar, este algoritmo no es práctico para redes grandes.

Buscando reducir el número de computaciones en el análisis multifractal en redes complejas, Liu y otros\cite{Liu2015} proponen una solución para redes complejas basado en el algoritmo propuesto para objetos geométricos de Tel y otros\cite{Tl1989}. Esta solución, la cual denominan como caja de arena, busca reducir el número de computaciones al no tener que calcularse el número mínimo de cajas de un radio determinado que se requiere para cubrir la red, sin embargo, se requiere repetir el proceso varias veces para obtener una buena estimación de las dimensiones.

\subsection{Aplicaciones prácticas}

El análisis multifractal se aplica en diferentes áreas del conocimiento, a continuación algunos trabajos recientes por área:

\begin{itemize}
    \item En biología, el análisis multifractal es aplicado para analizar relaciones en diferentes ámbitos\cite{Barat2016}, como son las relaciones en ecosistemas, entre proteínas\cite{Wang2014}, etc. Un trabajo destacado, es el estudio del genoma humano realizado por Moreno y otros\cite{Moreno2011}, en el cual se caracteriza la estructura de los diferentes cromosomas, encontrando que se puede categorizar de acuerdo a baja, media o alta multifractalidad, la cual se ve traducida en términos de estabilidad genética, es decir resistencia a factores ambientales que puedan modificar el ADN. Con estudios de este tipo se puede caracterizar algunas particularidades genéticas.
    \item En economía, la variación de precios en los mercados se ha estudiado desde diferentes enfoques, pasando por análisis de promedios por ventajas o estadístico, sin embargo, estos métodos presentan problemas debido a que estas son series de tiempo\cite{Siokis2014}. Un enfoque que ha permitido describir el comportamiento de estas, ha sido el análisis multifractal, un ejemplo de ello son el análisis de la red de pagos de Estonia\cite{RendondelaTorre2017}, en este articulo se encuentra que hay un patrón multifractal en esta red de pagos, por lo que toda la economía de este país, presenta un comportamiento multifractal. También, un hallazgo interesante, es que los exponentes de masa tienen un valor muy elevado, lo que indica que la red es bastante irregular. Finalmente, se concluye que este tipo de redes pueden ser estudiadas desde su esqueleto, lo que reduce considerablemente el número de computaciones que se requieren para obtener las dimensiones fractales.
    \item En la comunicación entre especies, se ha encontrado que el habla y los sonidos de algunos animales presenta patrones multifractales, un caso son los sonidos que emiten las aves migratorias\cite{Roeske2018}. El análisis multifractal permite concluir que los sonidos muestran comportamientos globales que varían de acuerdo al tono o la longitud del canto.
    \item El análisis multifractal se utiliza para estudiar la estructura de diferentes redes que modelan el tráfico en redes de telecomunicaciones\cite{DinhDang2004}, sistemas de tráfico\cite{Vojak1994} y redes sociales\cite{Wei2017}. En estas redes reales, se ha encontrado patrones monofractales en redes que son de mundo pequeño y multifractales en las que presentan semejanzas con redes libres de escala.
\end{itemize}


\subsection{Herramientas para el análisis fractal y multifractal}

En la exploración del estado del arte no se evidencia la existencia de una herramienta comercial para usuario final para el análisis multifractal. Lo que se encuentra son implementaciones realizadas por los investigadores de este tema y un complemento para el software Matlab\cite{matlab}. En la tabla \ref{tab:herramientasFractal} se pueden observar algunas de ellas:


\begin{longtable}{|p{3cm}|p{4cm}|p{2cm}|p{2cm}|p{4cm}|}
    \hline
    \textbf{Nombre} & \textbf{Descripción} & \textbf{Tipo de acceso} & \textbf{Lenguaje de programación} & \textbf{Funciones que provee} \\
    \hline
    \endhead
    Multifractal.jl\cite{multifractaljl} & Librería para el análisis multifractal de series de tiempo & Paga\footnote{La librería es gratuita, pero Matlab\cite{matlab} es un software pago \label{foot:matlab}} & Matlab & Determinar el espectro multifractal y dimensiones fractales.  \\
    \hline
    Multifractal Tool\cite{zenodo} & Herramienta para el análisis multifractal de series de tiempo y de imágenes & Paga\footnotemark[1] & Matlab &Provee análisis multifractal utilizando el espectro calculado con la transformada Wavelet, así mismo provee la reconstrucción multifractal de series de tiempo e imágenes. \\
    \hline
    Multifractal estimation using a standard box-counting algorithm\cite{lsaravia} & Esta librería permite realizar un análisis multifractal de un conjunto de imágenes & Libre & R y C++ & Permite calcular los exponentes de masa $T_q$ y dimensiones fractales $D_q$. Es trabajado en el articulo Multifractal analysis of spatial patterns in ecological communities\cite{Saravia2014}. \\
    \hline
    Multifractal Analysis ToolBox\cite{toolbox} & Complemento de Simulink para el análisis multifractal de señales (series e imágenes) & Paga & Matlab & Provee la gráfica de dimensión fractal, el cual denominan espectro multifractal. \\
    \hline
    Multiscale Multifractal Analysis (MMA)\cite{MMA} & Librería de análisis multifractal del Banco de Datos Physionet & Paga\footnotemark[1] & Matlab & Implementación del método propuesto en el articulo Multiscale multifractal analysis of heart rate variability recordings with a large number of occurrences of arrhythmia\cite{Gieratowski2012} \\
    \hline
    MDFA in Python\cite{pythonMFDA} & Modulo de Python para el análisis multifractal de series de tiempo e imágenes & Libre & Python & Esta es la implementación utilizada en al artículo Multifractal analysis for all. Frontiers in Physiology\cite{Jurica2015} \\
    \hline
    Implementation in Python of analysis multifractal\cite{pythonFractal} & Módulo en Python para análisis fractal de redes complejas & Libre & Python & Son las implementacion de los métodos propuestos en How to calculate the fractal dimension of a complex network- the box covering algorithm\cite{Song2007}.\\
    \hline
\caption{Herramientas para el análisis multifractal encontradas en la literatura}
\label{tab:herramientasFractal}
\end{longtable}

Se puede evidenciar en la exploración realizada, que no se encuentra una herramienta para el análisis multifractal en redes complejas.
\section{Resumen y conclusiones del capitulo}

En este capitulo se ha dado un marco referencial para entender el concepto de red compleja y así poder abordar el análisis de multifractalidad y robustez. También, se ha mostrado cómo se encuentra el estado del arte en estos temas. A continuación, algunos comentarios:

\begin{itemize}
    \item El análisis de redes complejas utiliza las herramientas que provee la teoría de grafos. Las propiedades estructurales y la clasificación de las redes complejas está basada con las que comúnmente se trabajan con grafos.
    \item En el estado del arte, el análisis fractal y multifractal es basado en métodos geométricos de cobertura de cajas, haciendo un mapeo espacial basado en la distancia más corta entre cada par de nodos.
    \item Así mismo, el análisis de robustez es el estudio de cómo cambian las propiedades estructurales de la red a medida que esta pierde nodos o conexiones, el significado de estos valores depende directamente de lo que se está modelando con la red.
    \item Finalmente, en la literatura se encuentra que la relación entre la medida de fractalidad y robustez no ha sido ampliamente explorada, por lo que realizar un estudio en esta materia, permite generar un aporte teórico para la generación de nuevas métricas y algoritmos.
\end{itemize}
\newpage
\section{Trabajos futuros}

\begin{itemize}
    \item Debe establecerse un método analístico para validar el análisis multifractal de los diferentes métodos, como se vio en las diferentes pruebas los resultados pueden variar significativamente
    \item El algoritmo de SandBox propuesto por Liu\cite{Liu2015}, es el que mejor resultados presenta en las pruebas realizadas, sin embargo, no se puede asegurar tenga mejor rendimiento que las técnicas de Inteligencia Artificial propuestas, por lo que se requiere un estudio a profundidad de este tema.
    \item Se debe proponer una estrategia para generar una heurística que permita seleccionar los centros de las cajas, ya que la propuesta de este trabajo depende si la red tiene estructura libre de escala, ya que se buscan nodos altamente conectados.
    \item A partir del trabajo realizado, una idea interesante a explorar son los generadores de redes monofractales, los cuales deben conservar su estructura ante diferentes estrategias de ataque
    \item Las pruebas de este trabajo deben ejecutarse en un entorno con mayor capacidad de cómputo, ya que los resultados encontrados en algunos casos no corresponden a lo esperado, ya que los algoritmos requieren iterar una gran cantidad de veces para obtener resultados cercanos a los algoritmos presentes en la literatura.
    \item Se debe proponer estrategias de distribución de los cálculos, ya que el análisis multifractal depende de la configuración de cajas a un radio dado $r$, este valor puede ser distribuido, ya que no hay dependencias entre ellos. Así mismo, para la robustez se puede probar con diferentes porcentajes de perdidas de nodos y realizar el respecto análisis multifractal.
\end{itemize}













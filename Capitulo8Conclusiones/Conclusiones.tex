\section{Conclusiones}

\begin{enumerate}
    \item Las redes de mundo pequeño muestran una estructura monofractal y conservan su estructura a media que son atacadas, esto debido a que son compactas. Esto es un resultado interesante, ya que además de la medida de distancia promedio de las distancias más cortas entre los nodos, se puede utiliza el análisis de multifractal y de robustez para caracterizar estas redes.
    \item En general, las estrategias de inteligencia artificial permiten estimar centros de cajas de las redes, lo que permite mejorar el tiempo de cómputo de los algoritmos BCC y FSBC, sin embargo tienen problemas de precisión, ya que encontrar los centros es un proceso iterativo que depende del número de nodos y la forma en que se interconectan. AL aumentar el número de iteraciones de los algoritmos se mejora la precisión, sin embargo, el costo computacional debe ser considerado, ya que los cálculos podrían tomar una gran cantidad de tiempo.
    \item La relación entre la robustez y la multifractalidad provee una medida que permite estudiar cómo se comporta la estructura interna de la red a medida que esta pierde nodos. Esto provee un marco de estudio para trabajos posteriores puedan usar esta relación para caracterizar redes reales como libres de escala o de mundo pequeño. 
    \item El uso de librerías, cuya representación interna es con matrices dispersas provee un Framework para el desarrollo de algoritmos para el análisis de multifractalidad, ya que al reducirse la complejidad espacial y computacional, al no representar información innecesaria, se realizan las búsquedas dentro de las redes con mayor rapidez que con otras opciones disponibles.
\end{enumerate}
\section{Conclusiones}

\begin{enumerate}
    \item La topología de una red puede describirse con base a la diferencia entre la máxima dimensión fractal y la mínima dimensión fractal. Si esta medida crece indica la existencia de diferentes componentes en la red interconectados entre sí por un conjunto de nodos clave. Por el contrario, si esta disminuye, la red tiende a tener una estructura compacta o monofractal.
    \item Las redes de mundo pequeño muestran una tendencia a ser monofractales y conservan su estructura a media que son atacadas, esto es debido a que son compactas. En consecuencia, se puede utilizar el análisis de multifractal y de robustez para caracterizar estas redes.
    \item Las relación entre multifractalidad y robustez está dada que a medida una red pierde nodos, la dimensión fractal de las estructuras presentes en la red va tendiendo a 0 y a una característica monofractal. Esto indica una descomposición de las estructuras de la red; siendo el caso más relevante el de las redes libres de escala a medida que pierden los nodos con mayor grado.
    \item Las diferentes estrategias de análisis multifractal muestran variaciones en el cálculo de las dimensiones fractales, esto se debe a que los centros se establecen aleatoriamente, lo que puede producir diferentes resultado de acuerdo a la red. Sin embargo, este análisis permite caracterizar las redes de acuerdo a su estructura, ya que entre más pronunciada sea la curva de la dimensión fractal, más diferencias estructurales existen dentro de la red.
    \item Se encuentra que realizar el análisis de multifractalidad y robustez permite concluir que redes que presentan un gran número de componentes son las más que se adaptan ante fallas aleatorias, ya que las curvas de dimensión fractal presentan poca variación si se pierde un porcentaje pequeño de nodos (10\% o 20\%) aleatoriamente. En el caso de las redes que tienden a ser monofractales, lo que indica que se adaptan ante cualquier tipo de ataque, en otras palabras, toleran fallos de cualquier tipo, sin embargo, se debe realizar un protocolo de pruebas más rigurosas para dar un soporte a este resultado.
    \item En general, las estrategias de inteligencia artificial permiten estimar centros de cajas de las redes, lo que permite mejorar el tiempo de cómputo de los algoritmos BCC y BCFS, sin embargo, se encuentra problemas de precisión, ya que la elección de los centros de las cajas es un proceso de búsqueda que depende del número de nodos y la forma en que se interconectan. Al aumentar el número de iteraciones de los algoritmos se mejora la precisión, sin embargo,los cálculos podrían tomar una gran cantidad de tiempo.
    \item Las estrategias de inteligencia artificial dan un indicio de que se obtienen mejores tiempos de ejecución de los algoritmos BCC y BCFS en el caso del análisis multifractal. Sin embargo, para el caso de SB los resultados no son concluyentes, debido a que las redes estudiadas no son los suficientemente grandes para llegar a un resultado concreto.
    \item Para el análisis de la robustez, los algoritmos de IA no mostraron ser una estrategia de ataque que produzca más efectos en la estructura de la red que las basadas en el grado y la centralidad de los nodos. Esto se debe a que en las redes estudiadas los nodos que tienen mayor centralidad o grado, son claves para la transmisión de información dentro de la red.
    \item El uso de librerías con representaciones comprimidas de las redes, provee un marco de trabajo para el desarrollo de algoritmos para el análisis de multifractalidad, ya que al reducirse la complejidad espacial y computacional se realizan las búsquedas dentro de las redes con mayor rapidez que con otras opciones disponibles.
    \item La herramienta desarrollada en este trabajo ofrece la posibilidad de realizar un análisis de multifractalidad y robustez de cualquier red compleja. Frente a las herramientas disponibles en la literatura se destaca que es libre, construida en un lenguaje flexible y común.Así mismo, cuenta con la optimización de algoritmos de inteligencia artificial, lo que permite realizar menos cómputos que los métodos de cubrimiento de cajas BC y BCFS.
    \item Las heurísticas desarrolladas en este trabajo deben ser mejoradas, sin embargo, se muestra que es posible diseñarlas para trabajar con redes complejas. La evolución de la función de evaluación muestra que es necesario realizar un proceso más amplio de exploración de opciones para mejorar la convergencia de los algoritmos de inteligencia artificial.
\end{enumerate}
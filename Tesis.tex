\documentclass[letterpaper, 11pt, oneside]{book}

\usepackage[utf8]{inputenc} %Para configuración de caracteres
\usepackage[spanish]{babel} %Para configuración de idioma

\usepackage{float} %Para que funcionen las tablas
\usepackage{anysize} %Para usar márgenes
%\marginsize{2cm}{2cm}{2cm}{2cm} %{izquierda}{derecha}{arriba}{abajo}. La superior esta sobre 1, la derecha sobre -0.5 y la de abajo sobre 2 (por el número de página)
%tablas
\usepackage{fancyhdr}
%tablas
\usepackage{booktabs}
\usepackage{longtable}
%rotar tablas (y usar \includegraphics )
\usepackage{rotating}
%Dividir en diagonal
%\usepackage{slashbox}
%color tablas
\usepackage{colortbl}
%colocar tabla en lugar de cuadro

\usepackage[numbib,notlof,notlot,nottoc]{tocbibind} %Para que la biliografía se llame así y no referencias y para que quede numerada como sección.
%\usepackage[notlof,notlot,nottoc]{tocbibind} %Bibliografía sin numerar

\bibliographystyle{ieeetr} %Estilo de bibliografía IEEE

\usepackage{url} % inserción url's notas de pie.

%Para poder colocar texto en color
\usepackage{color}
\definecolor{naranja}{rgb}{1,0.5,0} % valores de las componentes roja, verde y azul (RGB)
\definecolor{rojo}{rgb}{1,0,0}
\definecolor{SteelBlue}{rgb}{0.3,0.5,0.7}

\definecolor{dkgreen}{rgb}{0,0.6,0}
\definecolor{gray}{rgb}{0.5,0.5,0.5}
\definecolor{mauve}{rgb}{0.58,0,0.82}
\definecolor{darkblue}{rgb}{0.0,0.0,0.6}
\definecolor{cyan}{rgb}{0.0,0.6,0.6}
\definecolor{claregreen}{RGB}{4,180,95}
\usepackage{listingsutf8}
\lstset{ %
  basicstyle=\footnotesize,           % the size of the fonts that are used for the code
  numberstyle=\footnotesize,          % the size of the fonts that are used for the line-numbers
  numbersep=4pt,                  % how far the line-numbers are from the code
  backgroundcolor=\color{white},      % choose the background color. You must add \usepackage{color}
  breaklines=true,                % sets automatic line breaking
  breakatwhitespace=false,        % sets if automatic breaks should only happen at whitespace
  title=\lstname,                   % show the filename of files included with \lstinputlisting;{}
  extendedchars=false,
  inputencoding=utf8, 
}


%Para tener los bookmarks del pdf (menú al lado derecho en los visualizadores de pdf)
% si colorlinks= true, no salen las cajas, sino el color del link!!
% linkcolor para los indices, citecolor para las citas en texto, urlcolor para los enlaces
\usepackage[pdftex,bookmarks=true, pdfauthor={Grupo de Investigación en Inteligencia Artificial (GUIA)}, linkcolor=black, citecolor=black, colorlinks=true, urlcolor=black]{hyperref}
           

%Para poder hacer subfiguras (subfloat)
\usepackage{subfig}

\usepackage{enumitem} % Para poder continuar enumerate en otras partes

\usepackage{pdflscape}%Para colocar páginas horizontales en el PDF
\usepackage{multicol}



\begin{document}
	
\renewcommand{\tablename}{Tabla}
%%%Portada%%%
\input{Capitulo0Inicio/Portada.tex}


\renewcommand{\thepage}{\roman{page}}
%%%%%%Jurados%%%%%
\input{Capitulo0Inicio/Jurados}

%%%%%%%%Agradecimientos%%%%%%%
\newpage
\input{Capitulo0Inicio/Agradecimientos}


%%%%%%%%%%Tabla de contenido%%%%%%
\renewcommand{\contentsname}{Tabla de Contenido}
\tableofcontents

\renewcommand{\listfigurename}{Lista de Figuras}
\listoffigures 

\renewcommand{\listtablename}{Lista de tablas}
\listoftables

%%%Resumen%%%%%%
\chapter*{Resumen}
\label{capResumen}
\addcontentsline{toc}{chapter}{\nameref{capResumen}}

En este proyecto se plantea realizar la medición de parámetros para el análisis de multifractalidad y robustez en redes complejas mediante el uso de diferentes técnicas de inteligencia artificial. El objetivo del trabajo, es utilizar diferentes técnicas de inteligencia artificial para encontrar patrones o estrategias de búsqueda dentro del proceso de cálculo de las medidas, buscando reducir el espacio búsqueda de soluciones, entendiendo que la búsqueda en dicho espacio es un problema NP-Hard [42]. Finalmente, se desarrollará una librería que se pueda integrar a herramientas enfocadas en el estudio de redes complejas.
\\\\
El análisis de fractalidad y multifractalidad en redes complejas es útil en ciertas áreas del conocimiento, por ejemplo, en la astronomía se han encontrado patrones multifractales en algunos fenómenos físicos como el viento solar[26], o en las ciencias sociales donde se han
encontrado evidencia de estructuras multifractales en el comportamiento de los mercados[7].
\\\\
Las redes complejas que modelan sistemas reales suelen ser de gran tamaño[1] y los algoritmos existentes requieren realizar una o varias iteraciones en las redes para hallar estas mediciones[42], por ello en este trabajo, se estudiarán estrategias de búsqueda automática de
patrones o regularidades en dicho proceso con el fin de reducir el tiempo de cómputo
\\\\
En la literatura, se encuentran técnicas para la medición de la fractalidad, multifractalidad y robustez en redes complejas, como las técnicas de Box Covering[39] para fractalidad y las basadas parámetros en redes complejas para la robustez[28]. Por lo tanto, se plantea si desarrollar algoritmos basados en técnicas de inteligencia artificial para la búsqueda e identificación automática de patrones, permite realizar un cálculo eficiente de las mediciones de multifractalidad y robustez.




\thispagestyle{empty}


%%%%%%%%%%%%%%Introducción%%%%%%
\chapter*{Introducción}
\label{capIntro}
\addcontentsline{toc}{chapter}{\nameref{capIntro}}
\input{Capitulo0Inicio/Introduccion}
\thispagestyle{empty}

%%%%%Capitulo UNO: Contexto y objetivos%%%%%%%%%%%%%%%%%%%%%%%

\chapter{Contexto y objetivos}
\markboth{Contexto y objetivos}{Contexto y objetivos}
\renewcommand{\thepage}{\arabic{page}}
\setcounter{page}{1}
%%%%Formulación del problema%%%%%%%
\input{Capitulo1ContextoYObjetivos/Formulacion}

%%%%Objetivos%%%%
\input{Capitulo1ContextoYObjetivos/Objetivos}


%%%%%%%Alcances%%%%%%%
En este proyecto de grado se busca obtener una propuesta de solución al problema de gestión del espectro radioeléctrico usando programación con restricciones, con una aplicación prototipo en la que se puedan realizar desarrollos a futuro, por esta razón no se busca obtener un producto final.
\\ \\
Por lo tanto se han aplicado algunas limitaciones al desarrollo de la aplicación, para permitir su concepción, diseño e implementación dentro del contexto y alcance de un trabajo de grado de pregrado.

\section{Alcances metodológicos}	

La gestión del espectro se refiere a la asignación, actualización y remoción de la asignación de frecuencias para los operadores en el espectro radioeléctrico.  En la aplicación prototipo no se cambiarán ni eliminarán manualmente las bandas que han sido asignadas a partir de una solución que se ha obtenido automáticamente en el aplicativo.
\\ \\
No se va trabajar el estudio de interferencias; se da por descontado una distancia de $sep$ canales entre dos operadores dentro de una banda para garantizar no se interfieran entre sí.
\\ \\
La asignación es jerárquica en orden territorial por ejemplo, un canal asignado a un operador a nivel nacional está también asignado en cada región, departamento y municipio del país.
\\ \\
La asignación por entes territoriales no considera en detalle la asignación en sus divisiones por ejemplo, para el caso de un departamento no se toma la asignación en detalles de sus municipios, sino qué canales se encuentran ocupados en uno o más municipios. Por lo tanto una solución óptima en un ente territorial no lo es necesariamente en sus divisiones. Esta decisión se tomó por el gran tamaño que tomaban las entradas para describir el detalle de la asignación en cada división de un ente territorial.
\\ \\
En la práctica se establecen topes de espectro para cada operador en una banda, éste tope está definido en Hz, para efectos de simplificar, se toma un número de canales por banda, es de anotar que cada canal tiene un tamaño en Hz establecido por su frecuencia inicial y final.
\section{Alcances prácticos}

Con respecto al proceso de asignación se toman las siguientes consideraciones en el diseño de la aplicación:

\begin{itemize}
	\item El acceso a los datos utilizados en la práctica de la división territorial y asignación de frecuencias es restringido, por lo que en el proyecto se utilizan datos estimados.
	\item El modelo de los canales es variable en el cuadro nacional de atribución de frecuencias, en este trabajo de grado solamente se supone un modelo con una frecuencia de transmisión $T_{x}$ y una de recepción $R_{x}$.
	\item Se trabaja entre las frecuencias entre 26 MHz y 60G GHz, al omitir las frecuencias de servicio móvil marítimo y servicio fijo marítimo y las frecuencias más altas que no se encuentran canalizadas.
	\item Solamente se trabaja con las bandas de frecuencias definidas por servicios, las cuales se encuentran canalizadas, se ignora la información sobre la estructura de las bandas definidas por la ITU. Estas bandas se les denomina en el proyecto y aplicativo como rangos de frecuencia.
	\item La zonas territoriales del Colombia se dividen en entes territoriales, departamentos y municipios.
	\item La base de datos sólo contiene la información relevante para el proyecto, los datos de rangos de frecuencia y servicios han sido extraídos del cuadro nacional de atribución de frecuencias.
\end{itemize}



%%%%%%%Justificación%%%%%%%
\section{Justificación}

\subsection{Justificación teórica}

Este trabajo busca integrar el área de inteligencia artificial (IA) con el estudio de redes
complejas, las técnicas de inteligencia artificial ofrecen soluciones basadas en la identificación
de patrones y aprendizaje de problemas, que permiten diseñar soluciones que realizan una
búsqueda eficiente en problemas computacionalmente NP-Hard.
\\\\
Los algoritmos disponibles en la literatura para la medición de multifractalidad y robustez,
como el algoritmo de Box Covering, a medida que exploran la red, requieren la evaluación
de un conjunto de soluciones, lo que implica realizar cálculos en un espacio combinatorio.
Como consecuencia, no son prácticos para redes gran tamaño, por esta razón, en este trabajo
se aborda el uso de técnicas de inteligencia artificial para encontrar patrones o reglas que
permitan plantear técnicas para realizar estas mediciones más rápidamente.
\\\\
La identificación de patrones y regularidades en la medición de parámetros en redes complejas, ofrece una experiencia para el diseño de algoritmos y estrategias para el estudio de
características de diferentes sistemas complejos en diferentes áreas del conocimiento, como es
el caso de las ciencias sociales, donde las poblaciones pueden ser caracterizadas o clasificadas
como comunidades a través de una medida de modularidad, o en el estudio de redes de relaciones entre personas definidas por redes de conocidos o citaciones. Otro caso, es el estudio de
sistemas urbanos, como son las redes de transporte, en donde se pueden identificar estructuras
que ayuden a tomar decisiones para enfrentar problemas como los atascos de tráfico.

\subsection{Justificación metodológica}






%%%%%%%%%%%%%%%%%%%%%%%%%%%%%%Información sobre los capitulos%%%%%%%%%%%%%%%%%%%%%%%%%%%%%%%%%%%%%%%%%%%%%
%\input{Capitulo1ContextoYObjetivos/Informacion}

%%%%%%Capitulo DOS: Marco teórico%%%%%

\chapter[Revisión de literatura y Estado del arte]{Revisión de literatura y Estado del arte}
\markboth{Revisión de literatura y Estado del arte}{Revisión de literatura y Estado del arte}


\input{Capitulo2EstadoDelArte/Glosario.tex}
\input{Capitulo2EstadoDelArte/FractalidadYMultifractalidad.tex}
\input{Capitulo2EstadoDelArte/Robustez.tex}
\section{Algoritmos de inteligencia artificial en redes complejas}


%%%%%%%Capitulo TRES: Medición de multifractalidad%%%%%%

\chapter{Análisis de multifractalidad}\label{cap3}
\markboth{Análisis de multifractalidad}{Análisis de multifractalidad}

\input{Capitulo3Multifractalidad/BoxCounting.tex}
\input{Capitulo3Multifractalidad/AlgoritmoDeSandBox.tex}
\input{Capitulo3Multifractalidad/EstategiasEvolutivas.tex}
\input{Capitulo3Multifractalidad/EstrategiasSimulatedAnnealing.tex}

%%%%%Capitulo CUATRO: Análisis de Robustez%%%%%%%%
\chapter{Análisis de la robustez}
\markboth{Análisis de la robustez}{Análisis de la robustez}

\section{Análisis de robustez}

\subsubsection{Medidas de robustez}


\subsubsection{Estrategias de simulación de ataques}
\input{Capitulo4Robustez/EstrategiasEvolutivas.tex}
\input{Capitulo4Robustez/EstrategiasSimulated.tex}


%%%%%Capitulo CINCO: Implementacion%%%%%%%%
\chapter{Implementación de la librería} \label{capimp}
\markboth{Implementación de la librería}{Implementación de la librería}

\section{Desarrollo de la librería}

\subsection{Procesamiento de redes con SNAP}

\subsubsection{¿Porque SNAP?}

\subsubsection{Funciones utilizadas}

\subsection{Desarrollo de las funciones}



\subsection{Arquitectura de la librería}
\section{Requerimientos de la librería}


\subsection{Dependencias de software}

\subsection{Recomendaciones de capacidad de cómputo}
\section{Funciones provistas}

\subsection{Análisis multifractal}

\subsection{Análisis de robustez}

\subsection{Algoritmos evolutivos}

\subsection{Algoritmos de inteligencia artficial}
\section{API}

\subsection{Desarrollo del API}


\subsection{Recomendaciones de uso}



%%%%%%Capitulo SEIS: Análisis comparativo%%%%%%%%%%%%%%%%%%%%%
\chapter{Análisis comparativo}\label{cap6}
\markboth{Análisis comparativo}{Análisis comparativo}

\section{Redes utilizadas}

\subsection{Redes generadas a partir de modelos}


\subsection{Redes reales}

\input{Capitulo6Analisis/AnalisisMultifractal.tex}

\input{Capitulo6Analisis/AnalisisRobustez.tex}


%%%%CAPITULO 7: Conclusiones y trabajos futuros

\chapter{Conclusiones y trabajos futuros}\label{cap7}
\markboth{Conclusiones y trabajos futuros}{Conclusiones y trabajos futuros}

\section{Conclusiones}

\begin{enumerate}
    \item La topología de una red está describa en la diferencia entre la máxima dimensión fractal y la mínima dimensión fractal, si esta diferencia crece es un indicador de la existencia de diferentes componentes en la red interconectados entre sí por un conjunto de nodos clave, así mismo si este indicado disminuye, la red tiende a tener una estructura compacta.
    \item Las redes de mundo pequeño muestran una estructura monofractal y conservan su estructura a media que son atacadas, esto debido a que son compactas. Esto es un resultado interesante, ya que además de la medida de distancia promedio de las distancias más cortas entre los nodos, se puede utilizar el análisis de multifractal y de robustez para caracterizar estas redes. En las pruebas realizadas, evidencia que en las estrategias de ataque por centralidad, se muestra un efecto en la estructura de la red.
    \item Las relación entre multifractalidad y robustez está dada, que a medida una red pierde nodos, la dimensión fractal de las estructuras presentes en la red va tendiendo a 0 y a una característica monofractal. Esto indica una descomposición de las estructuras de la red; siendo el caso más relevante el de las redes libres de escala a medida que pierden los nodos con mayor grado.
    \item Las diferentes estrategias de análisis multifractal muestran variaciones en el cálculo de las dimensiones fractales, esto se debe a que los centros se establecen aleatoriamente, lo que puede producir diferentes resultado de acuerdo a la red. Sin embargo, permite caracterizar las redes de acuerdo a su estructura, ya que entre más pronunciada sea la curva de la dimensión fractal, más diferencias estructurales existen dentro de la red.
    \item Se encuentra que realizar el análisis de multifractalidad y robustez permite concluir que redes que presentan un gran número de componentes son las más que se adaptan ante fallas aleatorias, ya que las curvas de dimensión fractal presentan poca variación si se pierde un porcentaje pequeño de nodos (10\% o 20\%). En el caso de las redes que tienden a ser monofractales, son las que mejor se adaptan ante cualquier tipo de ataque, ya que el análisis multifractal permite intuir que su estructura interna es compacta, es decir toleran fallos de cualquier tipo, sin embargo, esto debe ser ampliado con la realización de pruebas más rigurosas.
    \item En general, las estrategias de inteligencia artificial permiten estimar centros de cajas de las redes, lo que permite mejorar el tiempo de cómputo de los algoritmos BCC y BCFS, sin embargo, tienen problemas de precisión, ya que encontrar los centros es un proceso iterativo que depende del número de nodos y la forma en que se interconectan. Al aumentar el número de iteraciones de los algoritmos se mejora la precisión, sin embargo, el costo computacional debe ser considerado, ya que los cálculos podrían tomar una gran cantidad de tiempo.
    \item Las estrategias de inteligencia artificial demostraron mejores tiempos de ejecución de los algoritmos BCC y BCFS en el caso del análisis multifractal. Sin embargo, para el caso de SB los resultados no son concluyentes, debido a que las redes estudiadas no son los suficientemente grandes para llegar a un resultado concreto.
    \item Para el análisis de la robustez, los algoritmos de IA no mostraron ser una estrategia de ataque que produzca más efectos en la estructura de la red que las basadas en el grado y la centralidad de los nodos. Esto se debe a que en las redes estudiadas los nodos que tienen mayor valor de centralidad, son claves para la transmisión de información dentro de la red.
    \item El uso de librerías con representaciones comprimidas de las redes, provee un marco de trabajo para el desarrollo de algoritmos para el análisis de multifractalidad, ya que al reducirse la complejidad espacial y computacional se realizan las búsquedas dentro de las redes con mayor rapidez que con otras opciones disponibles.
    \item La herramienta desarrollada en este trabajo ofrece la posibilidad de realizar un análisis de multifractalidad y robustez de cualquier red compleja. Frente a las herramientas disponibles en la literatura se destaca que es libre, construida en un lenguaje flexible y común.Así mismo, cuenta con la optimización de algoritmos de inteligencia artificial, lo que permite realizar menos cómputos que los métodos de cubrimiento de cajas BC y BCFS.
    \item Las heurísticas desarrolladas en este trabajo deben ser mejoradas, sin embargo, se muestra que es posible diseñarlas para trabajar con redes complejas. La evolución de la función de evaluación muestra que es necesario realizar un proceso más amplio de exploración de opciones para mejorar la convergencia de los algoritmos de inteligencia artificial.
\end{enumerate}
\newpage
\section{Trabajos futuros}

\begin{itemize}
    \item Debe establecerse un método analístico para validar el análisis multifractal de los diferentes métodos, como se vio en las diferentes pruebas los resultados pueden variar significativamente
    \item El algoritmo de SandBox propuesto por Liu\cite{Liu2015}, es el que mejor resultados presenta en las pruebas realizadas, sin embargo, no se puede asegurar tenga mejor rendimiento que las técnicas de Inteligencia Artificial propuestas, por lo que se requiere un estudio a profundidad de este tema.
    \item Se debe proponer una estrategia para generar una heurística que permita seleccionar los centros de las cajas, ya que la propuesta de este trabajo depende si la red tiene estructura libre de escala, ya que se buscan nodos altamente conectados.
    \item A partir del trabajo realizado, una idea interesante a explorar son los generadores de redes monofractales, los cuales deben conservar su estructura ante diferentes estrategias de ataque
    \item Las pruebas de este trabajo deben ejecutarse en un entorno con mayor capacidad de cómputo, ya que los resultados encontrados en algunos casos no corresponden a lo esperado, ya que los algoritmos requieren iterar una gran cantidad de veces para obtener resultados cercanos a los algoritmos presentes en la literatura.
    \item Se debe proponer estrategias de distribución de los cálculos, ya que el análisis multifractal depende de la configuración de cajas a un radio dado $r$, este valor puede ser distribuido, ya que no hay dependencias entre ellos. Así mismo, para la robustez se puede probar con diferentes porcentajes de perdidas de nodos y realizar el respecto análisis multifractal.
\end{itemize}














%%%%%BIBLIOGRAFIA%%%%%%%%
\bibliography{Bibliografia}

%%%%%%%%%%%%%%%%%%%%%%%%%%%%%%%%%%%%%%%%%%%%%%%%%%%%%%%%%%%%%%%%%%%%%%%%%%%%%%%%%%%%%%%%%%%%%%%%%%%%%%%%%%%
%%%%%%%%%%%%%%%%%%%%%%%%%%%%%%%%%%%%%%%%%%%%%ANEXOS%%%%%%%%%%%%%%%%%%%%%%%%%%%%%%%%%%%%%%%%%%%%%%%%%%%%%%%%
%%%%%%%%%%%%%%%%%%%%%%%%%%%%%%%%%%%%%%%%%%%%%%%%%%%%%%%%%%%%%%%%%%%%%%%%%%%%%%%%%%%%%%%%%%%%%%%%%%%%%%%%%%%
%\chapter*{Anexos}
%\addcontentsline{toc}{chapter}{Anexos}
%\markboth{Anexos}{Anexos}


\end{document}
